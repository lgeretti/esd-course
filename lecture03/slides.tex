\section{Introduction}

\begin{frame}
\frametitle{Introduction}
\framesubtitle{What is a version control system (VCS)?}

\begin{block}{A software that allows to keep snapshots ("revisions") of a code base}
\begin{itemize}
\item Snapshots allow to split the development phase into small chunks with well-identified concerns
\item Previous snapshots can be consulted to help identify bugs, to restore previous code, etc.
\item Having well-identified milestones on the code base allows to compare and merge results from multiple developers
\end{itemize}
\end{block}

\end{frame}

\begin{frame}
\frametitle{Introduction}
\framesubtitle{No version control: typical scenarios of code development}

\begin{enumerate}
\item
\begin{block}{Overwriting the same source files over and over}
\begin{itemize}
\item You can't check why a certain bug has now surfaced in respect to a previous version
\item If another developer also works on a given file, identifying/importing/exporting changes can be difficult
\end{itemize}
\end{block}
\pause
\item
\begin{block}{Keeping numbered archives of the code base}
\begin{itemize}
\item Two developers must agree on a unique way of numbering, if they want to interact
\item You most likely need to archive the whole code base
\item Such archives inevitably end up thrown around your filesystem, possibly on multiple machines
\end{itemize}
\end{block}
\end{enumerate}

\end{frame}

\begin{frame}
\frametitle{Introduction}
\framesubtitle{What does a version control system offer?}

\begin{block}{Essentially:}
\begin{itemize}
\item To {\em commit} a snapshot, when desired, along with a comment 
\item To create {\em branches} of development where experimental features can be tested in isolation
\item To compare entire commits or specific files
\item To {\em merge} commits, either between different branches or between different developers
\end{itemize}
\end{block}

\end{frame}

\begin{frame}
\frametitle{Introduction}
\framesubtitle{Centralized versus distributed VCS}

\begin{block}{Centralized}
\begin{itemize}
\item Commits are directly sent to a central {\em repository} from where all developers get their code base
\end{itemize}
\end{block}

\begin{block}{Distributed}
\begin{itemize}
\item Each developer has its local copy of a {\em remote} repository, and commits remain local until {\em pushed} remotely
\end{itemize}
\end{block}

See also: \\  \url{http://en.wikipedia.org/wiki/Comparison_of_revision_control_software}.

\end{frame}

\begin{frame}
\frametitle{Introduction}
\framesubtitle{Advantages of a distributed VCS}

\begin{itemize}
\item You can work from any place, without the need of network connectivity until you need to "publish" your commits
\item A local copy of the repository means maximum speed of comparisons, checkout of branches, etc.
\item The distinction between a {\em commit} and a {\em push} encourages having frequent, small commits that allow to {\em amend} commit mistakes and facilitate merging
\end{itemize}

\end{frame}

\section{Git}

\begin{frame}
\frametitle{Git}
\framesubtitle{Why Git?}

\begin{block}{Git is not the only DVCS out there}
\begin{itemize}
\item Open-source alternatives: notably Mercurial and Bazaar
\item Proprietary alternatives: notably BitKeeper
\end{itemize}
\end{block}
\pause
\begin{block}{Git has a growing spread}
\begin{itemize}
\item A lot of projects, both large and small, already use it
\item Many online code hosting services support it: e.g., Google Code, GitHub and BitBucket
\item It is integrated (natively or by plugins) into several IDEs
\item It is well known in the online community, documented and supported
\end{itemize}
\end{block}

\end{frame}