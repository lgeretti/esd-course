\section{Overview}

\begin{frame}
\frametitle{Overview}
\framesubtitle{What is SystemC?}

\begin{block}{SystemC is an HDL: Hardware Description Language}
Differently from other HDLs, its programming model facilitates integration with software by using C++ as a foundation language.
\end{block}

\end{frame}

\begin{frame}
\frametitle{Overview}
\framesubtitle{What SystemC adds above C++}

\begin{block}{Some hardware-oriented features:}
\begin{itemize}
\item Time model: time must have units, need clocks
\pause
\item Hardware data types: signals need high impedance and undefined state
\pause
\item Module hierarchy: structural hardware description requires components
\pause
\item Communication management: modules exchange data with different transports
\pause
\item Concurrency model: real-time communication and execution must be simulated properly
\end{itemize}
\end{block}

\end{frame}

\section{Elements}

\begin{frame}
\frametitle{Elements}
\framesubtitle{At a glance}

\begin{enumerate}
\item Modules and hierarchy
\item Threads and methods
\item Events, sensitivity and notification
\item Data types
\item Ports, interfaces and channels
\end{enumerate}

\end{frame}

\begin{frame}
\frametitle{Elements}
\framesubtitle{Modules and hierarchy}

\begin{block}{The \texttt{sc\_module} base class is used for modules}
It can be derived from to create a module, or a \texttt{SC\_MODULE} macro can be used.
\end{block}
\pause
\begin{block}{Modules are also containers: hierarchy}
A module may contain other modules, processes, channels and ports.
\end{block}
\pause
\begin{block}{A comparison to VHDL}
The entity goes into the header (.hpp) file, the architecture into the source (.cpp) file.
\end{block}

\end{frame}

\begin{frame}
\frametitle{Elements}
\framesubtitle{Threads and methods}

\begin{block}{A module is "run" by means of threads and methods}
Both are member functions of the module that are invoked as processes:
\begin{itemize}
\item \texttt{SC\_METHOD}: repeatedly run; no time passes between invocation and return of the function
\item \texttt{SC\_THREAD}: run once; may be suspended, allowing time to pass during execution
\end{itemize}
\end{block}

\end{frame}

\begin{frame}
\frametitle{Elements}
\framesubtitle{Events, sensitivity and notification}

\begin{block}{Processes communicate based on events}
The \texttt{sc\_event} and \texttt{sc\_event\_queue} classes are used.
\end{block}
\begin{block}{Methods and threads may have {\em sensitivity} to events}
SystemC offers {\em static} (i.e., defined at compile-time) sensitivity or {\em dynamic} (i.e., can be changed at run-time) sensitivity.

\medskip
Calls to \texttt{wait} and \texttt{notify} on a given event handle the actual communication between processes.
\end{block}

\end{frame}

\begin{frame}
\frametitle{Elements}
\framesubtitle{Data types}

\begin{block}{Data types are implemented using C++ templates}
Templates allow to {\em parameterize} the type, for example to set precision.
\end{block}
\pause
\begin{block}{Some relevant types:}
\begin{itemize}
\item \texttt{sc\_int}: an integer
\item \texttt{sc\_uint}: an unsigned integer
\item \texttt{sc\_fixed}: a fixed point non-integer
\item \texttt{sc\_logic}: a logical value (1,0,X,Z)
\item \texttt{sc\_lv}: a logical vector
\end{itemize}
\end{block}

\end{frame}

\begin{frame}
\frametitle{Elements}
\framesubtitle{Ports, interfaces and channels}

\begin{block}{Channels+interfaces model communication between modules}
The \texttt{sc\_channel} base class is used, along with some \texttt{sc\_interface} classes.
\end{block}
\pause
\begin{block}{Examples:}
\begin{itemize}
\item Channels: \texttt{sc\_mutex}, \texttt{sc\_semaphore}, \texttt{sc\_signal}, \texttt{sc\_fifo}
\item Interfaces: \texttt{sc\_signal\_in\_if}, \texttt{sc\_signal\_inout\_if}, \texttt{sc\_fifo\_out\_if}
\end{itemize}
\end{block}
\pause
\begin{block}{Modules can be accessed through ports}
In this way, we can match module interfaces with channel interfaces.
\end{block}

\end{frame}

\section{Installation}

\begin{frame}
\frametitle{Installation of the library}
\framesubtitle{Get the sources}

\begin{block}{SystemC sources are under an Open Source License}
You need to do a free registration at 
{ \scriptsize
\url{http://www.accellera.org/kmembership_info/individual_signup/}
}

\medskip 
Then you can download \texttt{systemc-2.3.0.tgz} from 
{ \scriptsize
\url{http://www.accellera.org/downloads/standards/systemc}
}
\end{block}
\begin{block}{Extract the sources}
\texttt{\$ tar xf systemc-2.3.0.tgz} \\
will extract a \texttt{systemc-2.3.0} directory that you can move to, for example,
\texttt{\textasciitilde{}/sources/}
\end{block}

\end{frame}

\begin{frame}
\frametitle{Installation of the library}
\framesubtitle{Build and install}

\begin{block}{For build and local install}
\begin{enumerate}
\item \texttt{\$ mkdir build \&\& cd build}
\item \texttt{\$ ../configure}
\item \texttt{\$ make}
\item \texttt{\$ sudo make install}
\end{enumerate}
You can verify the correct installation with: \texttt{\$ make check}
\end{block}
\pause
\begin{block}{For global install}
\begin{enumerate}
\setcounter{enumi}{3}
\item \texttt{\$ sudo cp -R ../include/* /usr/include}
\item \texttt{\$ sudo cp ../lib-linux64/* /usr/lib}
\end{enumerate}
For 32 bits Linux machines use \texttt{lib-linux}, for 64 bits OSX machines use \texttt{macosx64}.
\end{block}

\end{frame}

\begin{frame}[fragile]
\frametitle{Installation of the library}
\framesubtitle{Check the global install}
{ \scriptsize 
\begin{verbatim}
#include <systemc.h>
#include <iostream>

SC_MODULE(Hello) {
  SC_CTOR(Hello) {
    SC_THREAD(salutation);
  }

  void salutation() {
    std::cout << "Hello SystemC world!" << std::endl;
  }
};

int sc_main(int sc_argc, char* sc_argv[]) {
  Hello hello("helloInstance");
  sc_start();
  return 0;
}
\end{verbatim}
\begin{itemize}
\item Compile with: \texttt{\$ g++ hello.cpp -o hello -l systemc}
\end{itemize}
}

\end{frame}
