\section{Introduction}

\begin{frame}
\frametitle{Introduction}
\begin{block}{What are ports?}
Ports are pointers to channels. They allow access to channels across module boundaries.
\end{block}
\pause
\begin{block}{Repository for some examples:}
\url{https://bitbucket.org/uniud_esd/systemc-ports}
\end{block}
\end{frame}

\section{Generic ports}

\begin{frame}[fragile]
\frametitle{Generic ports}
\framesubtitle{Declaration}
\begin{block}{A port is declared with \texttt{sc\_port<T>}}
\texttt{sc\_port<interface> portname;} \\
\medskip
The \texttt{interface} is an interface used by a channel. 
\begin{itemize}
\item Interfaces for a given channel type are
defined within \texttt{/usr/include/sysc/communication}
\end{itemize}
\end{block}
\pause
\begin{block}{Connection}
\begin{itemize}
\item Modules at the same level {\em bind} their ports via a channel;
\item A hierarchic module internally binds its ports directly to ports of its submodules.
\end{itemize}
\end{block}
\end{frame}

\begin{frame}[fragile]
\frametitle{Generic ports}
\framesubtitle{Implementation example}
{\scriptsize 
\begin{verbatim}
SC_MODULE(InnerModule) {

    sc_port<sc_fifo_in_if<unsigned> > din;
    sc_port<sc_fifo_out_if<unsigned> > dout;

    SC_CTOR(InnerModule) {
      SC_THREAD(elevate_thread);
    }

  private:
    void elevate_thread();
};

void InnerModule::elevate_thread() {
  while(true) {
    unsigned data = din->read();
    unsigned result = (1<<data); // 2^data
    dout->write(result);
  }
}
\end{verbatim}
}
\vspace{-1em}
\end{frame}

\begin{frame}[fragile]
\frametitle{Generic ports}
\framesubtitle{Connections}
{\scriptsize 
\begin{verbatim}
SC_MODULE(InnerModule) {
    sc_port<sc_fifo_in_if<unsigned> > din;
    sc_port<sc_fifo_out_if<unsigned> > dout;
    ...
};

SC_MODULE(OuterModule) {
    sc_port<sc_fifo_in_if<unsigned> > din;
    sc_port<sc_fifo_out_if<unsigned> > dout;
    sc_fifo<unsigned> dint; 
     
    InnerModule inner1, inner2;
    ...
};

OuterModule::OuterModule(sc_module_name nm) : 
  sc_module(nm), inner1("inner1"), inner2("inner2") {
  
    inner1.din(this->din);
    inner1.dout(this->dint);
    inner2.din(this->dint);
    inner2.dout(this->dout);
}

\end{verbatim}
}
\vspace{-1em}
\end{frame}

\section{Port arrays}

\begin{frame}[fragile]
\frametitle{Port arrays}
\framesubtitle{Declaration}

\begin{block}{Arrays must have a size and a binding policy}
\texttt{sc\_port<interface[,N[,POL]]> portname;}
\begin{itemize}
\item \texttt{N}: number of ports (defaults to 1, use 0 for unlimited);
\item \texttt{POL}: policy, decides how many ports must be bound; values:
\begin{enumerate}
\item \verb@SC_ONE_OR_MORE_BOUND@ (default)
\item \verb@SC_ZERO_OR_MORE_BOUND@
\item \verb@SC_ALL_BOUND@
\end{enumerate}
\end{itemize}
\end{block}
\begin{block}{Example}
\vspace{-0.5em}
{\scriptsize 
\begin{verbatim}
SC_MODULE(Switch) {
  sc_port<sc_fifo_in_if<int>,5,SC_ZERO_OR_MORE_BOUND> T1_ip;
  sc_port<sc_signal_inout_if<bool>,0> request_op;
  ...
};
\end{verbatim}
}
\vspace{-0.5em}
\end{block}
\end{frame}

\begin{frame}[fragile]
\frametitle{Port arrays}
\framesubtitle{Usage example}

{\tiny 
\begin{verbatim}
SC_MODULE(Board) {
  Switch switch_i;
  sc_fifo<int> t1A, t1B, t1C, t1D;
  sc_signal<bool> request[9];
  SC_CTOR(Board) : switch_i("switch_i") {
    switch_i.T1_ip(t1A);
    switch_i.T1_ip(t1B);
    switch_i.T1_ip(t1C);
    switch_i.T1_ip(t1D);
    for (unsigned i=0;i!=9;i++)
      switch_i.request_op(request[i]);
    ...
  }
  ...
};

void Switch::switch_thread() {
  for (unsigned i=0;i!=request_op.size();i++) 
    request_op[i]->write(true);
  wait(T1_ip[0]->data_written_event() | 
       T1_ip[1]->data_written_event() |
       T1_ip[2]->data_written_event() | 
       T1_ip[3]->data_written_event());
  while(true) {
    for (unsigned i=0;i!=T1_ip.size();i++) {
      int value = T1_ip[i]->read();
      ... // Process each port 
    }
  }
}
\end{verbatim}
}
\end{frame}

\section{Signal port specializations}

\begin{frame}[fragile]
\frametitle{Signal port specializations}
\framesubtitle{Declation and module construction}
\begin{block}{Specializations are clear and handy}
\vspace{0.5em}
\begin{itemize}
\item Input ports:
{\scriptsize 
\begin{verbatim}
sc_in<T> name_sig_ip;
sc_in_resolved name_sig_ip;
sc_in_rv<N> name_sig_ip;
sensitive << name_sig_ip;
sensitive << name_sig_ip.pos();
sensitive << name_sig_ip.neg();
\end{verbatim}
}
\vspace{0.5em}
\pause
\item Output ports:
{\scriptsize 
\begin{verbatim}
sc_inout<T> name_sig_op;
sc_inout_resolved name_sig_op;
sc_inout_rv<N> name_sig_op;
sensitive << name_sig_op;
sensitive << name_sig_op.pos();
sensitive << name_sig_op.neg();
name_sig_op.initialize(value);
\end{verbatim}
}
\end{itemize}

\end{block}

\end{frame}

\begin{frame}[fragile]
\frametitle{Signal port specializations}
\framesubtitle{Usage example}
\begin{block}{Function calls on underlying signals use the \texttt{->} operator}
\vspace{-0.5em}
{\scriptsize
\begin{verbatim}
SC_MODULE(LFSR) {
  sc_in<bool> sample, clock, reset;
  sc_out<sc_int<32> > signature;
  SC_CTOR(LFSR) {
    SC_METHOD(shift_method);
    sensitive << clock.pos() << reset;
    signature.initialize(0);
  }
  sc_int<32> LFSR_reg;
  void shift_method() {
    if (reset->read() == true) {
      LFSR_reg = 0;
      signature->write(LFSR_reg);
    } else {
      ...
    }
  }
};
\end{verbatim}
}
\vspace{-0.5em}
\end{block}
\end{frame}

\section{Exports}

\begin{frame}[fragile]
\frametitle{Exports}
\framesubtitle{What are exports?}
\begin{block}{Syntax: \texttt{sc\_export<interface> name;}}
Exports are ports that are internally bound to a channel.
\begin{itemize}
\item They do not need to expose the channel details (or even have an actual underlying channel, for that matter).
\end{itemize}
\end{block}
\pause
\begin{block}{Differences compared to ports}
\begin{enumerate}
\item Binding is done within the declaring module;
\item You cannot have export arrays;
\item You cannot use an export on a static sensitivity list.
\end{enumerate}
\end{block}

\end{frame}

\begin{frame}[fragile]
\frametitle{Exports}
\framesubtitle{What are exports?}
\begin{block}{Syntax: \texttt{sc\_export<interface> name;}}
Exports are ports that are bound to a channel {\em internal} to the module.
\begin{itemize}
\item They do not need to expose the channel details.
\end{itemize}
\end{block}
\pause
\begin{block}{Differences compared to ports}
\begin{enumerate}
\item Binding is done within the declaring module;
\item You cannot have export arrays;
\item You cannot use an export on a static sensitivity list.
\end{enumerate}
\end{block}

\end{frame}

\begin{frame}[fragile]
\frametitle{Exports}
\framesubtitle{Usage example}
{\tiny 
\begin{verbatim}
SC_MODULE(ExportInnerModule) {

    sc_port<sc_signal_inout_if<unsigned> > din;
    sc_export<sc_signal_inout_if<unsigned> > dout;
    sc_signal<unsigned> dout_ch;

    SC_CTOR(ExportInnerModule) : din("din"), dout("dout") {
      SC_THREAD(elevate_thread);
        sensitive << din;
      dout(dout_ch);
    }
    
    void elevate_thread();
};

SC_MODULE(ExportOuterModule) {

    sc_port<sc_signal_inout_if<unsigned> > din;
    sc_export<sc_signal_inout_if<unsigned> > dout;
    
    ExportInnerModule inner1, inner2;
    
    SC_CTOR(ExportOuterModule) : inner1("inner1"), inner2("inner2") {
        inner1.din(din);
        inner2.din(inner1.dout);
        dout(inner2.dout);
    }
};
\end{verbatim}
}

\end{frame}