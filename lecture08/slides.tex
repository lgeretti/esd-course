\section{Introduction}

\begin{frame}
\frametitle{Introduction}
\framesubtitle{C++ data types}

\begin{block}{You could use the following:}
\begin{itemize}
\item bool: true/false, but also binary data
\item char: ascii character, but also byte data
\item short: not shorter than one byte
\item int: not shorter than a short
\item long: not shorter than an int
\item long long: not shorter than a long
\item float: 32 bits floating point
\item double: 64 bits floating point
\item long double: not shorter than a double
\end{itemize}
But they depend on the architecture, hence are not portable.
\end{block}
\end{frame}

\begin{frame}
\frametitle{Introduction}
\framesubtitle{SystemC data types}

\begin{block}{SystemC has more flexible types}
They are (templated) classes:
\begin{itemize}
\item Their bit width is configurable
\item They still have operators
\item They also have some utility methods
\end{itemize}
\end{block}
\pause
\begin{block}{Categories:}
\begin{enumerate}
\item Bit vector
\item Logic
\item Integer
\item Fixed-point
\end{enumerate}
\end{block}

\end{frame}

\section{Categories}

\subsection{Bit vector}

\begin{frame}[fragile]
\frametitle{Categories}
\framesubtitle{Bit vector}

\begin{block}{\texttt{sc\_bv<BITWIDTH> NAME...;}}
\begin{itemize}
\item supports bit selection (i.e., {\bfseries \texttt{[]}}) and \texttt{range()}.
\item supports bitwise operations (i.e., \verb+&,|,^+)
\item has 1-bit reduction operations in the form: \texttt{and\_reduce()}, \texttt{or\_reduce()}, \texttt{xnor\_reduce()}, etc.
\end{itemize}
\end{block}
\pause
\begin{block}{Examples:} 
\vspace{-1em}
\begin{verbatim}
sc_bv<5> positions = "01101";
sc_bv<6> mask = "100111";
sc_bv<5> active = positions & mask; // gives 00101
sc_bv<1> all = active.and_reduce(); // gives 0
positions.range(3,2) = "00"; // changes to 00001
\end{verbatim}
\vspace{-1em}
\end{block}

\end{frame}

\subsection{Logic}

\begin{frame}[fragile]
\frametitle{Categories}
\framesubtitle{Logic}

{\scriptsize 
\begin{block}{Scalar and vector}
\begin{itemize}
\item \texttt{sc\_logic NAME...;}
\item \texttt{sc\_lv<BITWIDTH> NAME...;}
\end{itemize}
Values for the single element:
\begin{itemize}
\item 0: \,\, \texttt{SC\_LOGIC\_0, Log\_0, '0'}
\item 1: \,\, \texttt{SC\_LOGIC\_1, Log\_1, '1'}
\item Z: \,\, \texttt{SC\_LOGIC\_Z, Log\_Z, 'Z', 'z'}
\item X: \,\, \texttt{SC\_LOGIC\_X, Log\_X, 'X', 'x'}
\end{itemize}
Operators, ranges and reductions work as with bit vectors.
\end{block}
\pause
\begin{block}{Examples:} 
\vspace{-1em}
\begin{verbatim}
sc_lv<5> positions = "01xz1";
sc_lv<6> mask = "10ZX11";
sc_lv<5> active = positions & mask; // gives 0xxx1
active.range(0,0) = "z"; // changes to 0xxxz
sc_logic reduction = active.or_reduce(); // gives x
\end{verbatim}
\vspace{-1em}
\end{block}
}
\end{frame}

\begin{frame}
\frametitle{Categories}
\framesubtitle{Logic}

{\scriptsize 
\begin{block}{Logic tables}
\begin{table}
\begin{tabular}{|c|c|c|c|c|}
\hline
{\bfseries AND} & {\em 0} & {\em 1} & {\em X} & {\em Z}  \\
\hline
{\em 0} & 0 & 0 & 0 & 0 \\
\hline
{\em 1} & 0 & 1 & X & X \\
\hline
{\em X} & 0 & X & X & X \\
\hline
{\em Z} & 0 & X & X & X \\
\hline
\end{tabular}
\end{table}
\begin{table}
\begin{tabular}{|c|c|c|c|c|}
\hline
{\bfseries OR} & {\em 0} & {\em 1} & {\em X} & {\em Z}  \\
\hline
{\em 0} & 0 & 1 & X & X \\
\hline
{\em 1} & 1 & 1 & 1 & 1 \\
\hline
{\em X} & X & 1 & X & X \\
\hline
{\em Z} & X & 1 & X & X \\
\hline
\end{tabular}
\end{table}

\begin{table}
\begin{tabular}{|c|c|c|c|c|}
\hline
{\bfseries NOT} & {\em 0} & {\em 1} & {\em X} & {\em Z}  \\
\hline
 & 1 & 0 & X & X \\
\hline
\end{tabular}
\end{table}
\end{block}
}
\end{frame}

\subsection{Integer}

\begin{frame}[fragile]
\frametitle{Categories}
\framesubtitle{Integer}

\begin{block}{Signed/unsigned, normal or big}
\begin{itemize}
\item \texttt{sc\_int<BITWIDTH> NAME...;}
\item \texttt{sc\_uint<BITWIDTH> NAME...;}
\item \texttt{sc\_bigint<BITWIDTH> NAME...;}
\item \texttt{sc\_biguint<BITWIDTH> NAME...;}
\end{itemize}
You have better control compared to c++ native integers.
\end{block}
\pause
\begin{block}{Examples:} 
\vspace{-1em}
\begin{verbatim}
sc_int<5> small_signed=-3;
sc_uint<13> large(4000);
sc_uint<14> large_signed = large + small_signed;
sc_biguint<320> very_large;
\end{verbatim}
\vspace{-1em}
\end{block}

\end{frame}

\subsection{Fixed-point}

\begin{frame}
\frametitle{Categories}
\framesubtitle{Fixed-point}
{\scriptsize
\begin{block}{Fixed vs Fix}
\vspace{-1em}
\begin{itemize}
\item \texttt{sc\_fixed<WL,IWL[,QUANT[,OVFLW[,NBITS]]]> NAME;}
\item \texttt{sc\_ufixed<WL,IWL[,QUANT[,OVFLW[,NBITS]]]> NAME;}
\item \texttt{sc\_fixed\_fast<WL,IWL[,QUANT[,OVFLW[,NBITS]]]> NAME;}
\item \texttt{sc\_ufixed\_fast<WL,IWL[,QUANT[,OVFLW[,NBITS]]]> NAME;}
\end{itemize}
\begin{itemize}
\item \texttt{sc\_fix NAME(WL,IWL[,QUANT[,OVFLW[,NBITS]]]);}
\item \texttt{sc\_ufix NAME(WL,IWL[,QUANT[,OVFLW[,NBITS]]]);}
\item \texttt{sc\_fix\_fast NAME(WL,IWL[,QUANT[,OVFLW[,NBITS]]]);}
\item \texttt{sc\_ufix\_fast NAME(WL,IWL[,QUANT[,OVFLW[,NBITS]]]);}
\end{itemize}
\end{block}
\pause
\begin{block}{Fields:} 
\vspace{-1em}
\begin{itemize}
\item WL: total word length
\item IWL: integer word length
\item QUANT: quantization mode (optional)
\item OVFLW: overflow mode (optional)
\item NBITS: number of saturation bits (optional)
\end{itemize}
\vspace{-1em}
\end{block}
}
\end{frame}

\begin{frame}[fragile]
\frametitle{Categories}
\framesubtitle{Fixed-point}
{\scriptsize
\begin{block}{When to choose one over the other?}
\begin{itemize}
\item Fixed-point: set at compile-time, hence more efficient but also requires more compilation time
\item Fix-point: set at run-time during object creation, hence more flexible but less efficient
\end{itemize}
Also, \texttt{fast} types are faster but limited to 53 bit precision, since they use \texttt{double} internally.

\pause
\medskip
To use fixed-point types in your sources, you need to add a 

\medskip
\texttt{\#define SC\_INCLUDE\_FX}

\medskip
before including the SystemC headers. This is to avoid unnecessary compilation overhead when fixed-point types are not used.
\end{block}
\pause
\begin{block}{Examples:} 
\vspace{-1em}
\begin{verbatim}
const sc_ufixed<19,3> PI("3.141592654");
sc_fix oil_temp(20,17);
sc_fixed_fast<7,1> valve_opening;
oil_temp += PI;
\end{verbatim}
\vspace{-1em}
\end{block}
}
\end{frame}

\section{Helpers}

\subsection{Operators and methods}

\begin{frame}[fragile]
\frametitle{Helpers}
\framesubtitle{Operators and methods}
{\scriptsize 
\begin{block}{Operators}
\begin{itemize}
\item Comparison: \, \verb@== != > >= < <=@
\item Arithmetic: \, \verb@++ -- * / \ + - << >>@
\item Assignment: \, \verb@= &= |= ^=  *= /= %= += -= <<= >>=@
\end{itemize}
\end{block}
\pause
\begin{block}{Methods}
\begin{itemize}
\item Bit selection: \, \verb@bit(idx), [idx]@
\item Range selection: \, \verb@range(high,low), (high,low)@
\item Conversion to C++ types: \, \verb@to_double(), to_int(), to_int64(),@ \verb@to_long(), to_uint(), to_uint64(), to_ulong()@
\item Testing: \, \verb@is_zero(), is_neg(), length()@
\item Bit reduction: \, \verb@and_reduce(), nand_reduce(), or_reduce(),@ \verb@nor_reduce(), xor_reduce(), xnor_reduce()@
\end{itemize}
\end{block}
}
\end{frame}

\begin{frame}[fragile]
\frametitle{Helpers}
\framesubtitle{Type conversion}
{\scriptsize 
\begin{block}{You can convert between types}
No methods are required: just assign between types
\end{block}
\pause
\begin{block}{Examples}
\vspace{-1em}
\begin{verbatim}
sc_uint<3> myInt = 6;
sc_lv<4> a = myInt; // returns "0110"
myInt += 2;
sc_bv<3> b = myInt; // this now overflows, returns "000"
sc_bv<5> c = -1; // input is assumed as signed int, returns "11111" 
\end{verbatim}
\vspace{-1em}
\end{block}
}
\end{frame}

\subsection{Printing}

\begin{frame}
\frametitle{Helpers}
\framesubtitle{Printing}

\begin{block}{SystemC uses C++ streams}
You can output a value using \texttt{cout} normally.
\end{block}
\pause
\begin{block}{If you want more control, use \texttt{to\_string}}
Syntax:
\texttt{to\_string(sc\_numrep rep[, bool wprefix])}
\begin{table}
\begin{tabular}{|c|c|c|}
\hline
\texttt{sc\_numrep} & {\em Prefix} & {\em Meaning} \\
\hline
\texttt{SC\_DEC} & 0d & Decimal \\
\hline
\texttt{SC\_BIN} & 0b & Binary \\
\hline
\texttt{SC\_HEX} & 0x & Hexadecimal \\
\hline
\texttt{SC\_OCT} & 0o & Octal \\
\hline
\end{tabular}
\end{table}
where \texttt{wprefix} shows the representation prefix (true by default).
\end{block}
\end{frame}

\section{Considerations}

\begin{frame}[fragile]
\frametitle{Considerations}
\framesubtitle{Which types to choose}

\begin{block}{Use the types closest to C++ that you can}
This depends on the amount of simulation versus synthesis you are going to perform.
\end{block}
\pause
\begin{block}{Data type performance}
\vspace{-0.5em}
\begin{table}
\begin{tabular}{l|l}
\hline
Speed & Data type \\
\hline
Fastest & \verb@int, double, bool, unsigned int, etc.@ \\
        & \verb@sc_int, sc_uint@ \\
        & \verb@sc_bv, sc_logic, sc_lv@ \\
        & \verb@sc_bigint, sc_biguint@ \\
        & \verb@sc_fixed_fast, sc_fix_fast@ \\
        & \verb@sc_ufixed_fast, sc_ufix_fast@ \\
        & \verb@sc_fixed, sc_fix@ \\ 
Slowest & \verb@sc_ufixed, sc_ufix@ \\
\hline
\end{tabular}
\end{table}
\vspace{-1em}
\end{block}

\end{frame}

\section{Exercise}

\begin{frame}
\frametitle{Exercise}
\framesubtitle{Play with types}

\begin{block}{What to do:}
Perform operations using the four categories, employing methods where appropriate.
\end{block}
\begin{block}{Sample repository:}
\url{https://bitbucket.org/uniud_esd/systemc-types} 

\medskip
{\scriptsize (to clone: \texttt{\$ git clone git@bitbucket.org:uniud\_esd/systemc-types})}
\end{block}

\end{frame}
