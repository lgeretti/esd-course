\section{Introduction}

\begin{frame}
\frametitle{Introduction}
\framesubtitle{The need of tests}

\begin{block}{Software tests simplify checking the correctness of a design}
Without tests, we would defer such operation to the physical product.
\end{block}
\pause
\begin{block}{The special need for system design}
Systems are made of many components. If any one component is faulty, the whole system experiences issues. 
Therefore we can have the following:
\begin{itemize}
\item component-level tests to help you isolate localized issues;
\item system-level tests to check the interaction between components.
\end{itemize}
\end{block}
\end{frame}

\begin{frame}
\frametitle{Introduction}
\framesubtitle{Properties of a good test {\em suite}}

\begin{enumerate}
\item Coverage: tests address as much functionality as possible;
\pause
\item Atomicity: a single test checks only one specific state or aspect of the component;
\pause
\item Reproducibility: a single test has to return the same result for multiple runs;
\pause
\item Automation: failure/success of a test is verified programmatically.
\end{enumerate}

\end{frame}

\begin{frame}
\frametitle{Introduction}
\framesubtitle{The need of automation for a test suite}

\begin{block}{Complex systems may have lots of tests}
If no automation is used, we tend to eventually ignore some tests with the {\em false} assumption that they still pass. When automating, all tests can be executed with very little time cost and negligible user intervention.
\end{block}
\pause
\begin{block}{Properties of an automated test:}
\begin{itemize}
\item Inputs and expected outputs are included in the test;
\item Simply executed, returns success or failure;
\item Usually written for a specific testing framework (e.g., CMake); 
\item Still may have ancillary outputs (e.g., wave files, debugging info on the standard output) that
don't interact with the testing framework.
\end{itemize}
\end{block}
\pause

\end{frame}

\section{TDD}

\begin{frame}
\frametitle{Test-driven development (TDD)}
\framesubtitle{The methodology}

\begin{block}{Think about your design in terms of tests}
This methodology suggests that tests are created first, then the functionality is progressively added and refined to pass the tests.
\end{block}
\pause
\begin{block}{Approach:}
\begin{enumerate}
\item Write at least one test that will verify your component;
\begin{itemize}
\item Will not compile correctly yet, since the component does not exist!
\end{itemize}
\pause
\item Write an implementation for the component that would make the tests compile;
\pause
\item Fix the implementation until all tests pass;
\pause
\item Add new tests if appropriate;
\pause
\item Continue from 3 until satisfied with your test suite.
\end{enumerate}
\end{block}

\end{frame}

\begin{frame}
\frametitle{Test-driven development (TDD)}
\framesubtitle{An example: an Adder component}

This is a toy example that shows the rationale behind the methodology:
\begin{enumerate}
\item Write one test that requires 2+2 to be equal to 4; the Adder module is referenced but not implemented;
\pause
\item Write an "empty" Adder module that does nothing of significance, but makes the test compile;
\pause
\item Write an implementation that returns 4 always: this is the simplest implementation that satisfies the test suite;
\pause
\item Is the test suite satisfactory? No, because we should test for multiple inputs. So we add a new test for 7+1 equal to 8;
\pause
\item Now we are compelled to write a more generic (hence more usable) implementation, which is probably the correct one.
\end{enumerate}

\end{frame}

\begin{frame}
\frametitle{Test-driven development (TDD)}
\framesubtitle{Summarizing}

\begin{block}{What's the point of the methodology?}
Since the implementation is written for the tests, it's more difficult that some behavior of the component is not tested.
\end{block}
\pause
\begin{block}{The key requirement:}
You must provide sufficient tests (i.e., input/output sets) to represent the actual states in which the component may be. In other terms, you must reason in terms of how you expect the component to react to certain inputs.
\end{block}
\pause
\begin{block}{When are my tests sufficient?}
Sometimes you cannot simply know for sure. In other cases you can identify a limited set of component states to test.
\begin{itemize}
\item Covering all {\em border cases} is a good approach. This usually translates into providing {\em extremal} values for inputs.
\end{itemize}
\end{block}
\end{frame}

\section{Testing in SystemC}

\begin{frame}
\frametitle{Testing in SystemC}
\framesubtitle{Testbenching}

\begin{block}{A testbench is an environment for testing}
It is a module that includes the module under test (MUT), along with a {\em stimulus} and a {\em checker} thread.
\end{block}
\pause
\begin{block}{How do we test in a testbench?}
\begin{enumerate}
\item Create a stimulus thread that provides inputs for the MUT;
\pause
\item Create a checker thread that collects outputs from the MUT, verifies their values and saves the success on a variable;
\pause
\item Simulate the testbench module;
\pause
\item Read the success variable to get the result of the test.
\end{enumerate}
\end{block}

\end{frame}

\begin{frame}
\frametitle{Testing in SystemC}
\framesubtitle{Stimulus}

\begin{block}{The stimulus writes the inputs}
We usually employ a SystemC thread, since we need to use {\em wait} conditions.
\end{block}
\pause
\begin{block}{What if the combination of inputs is too large to be fully tested?}
Test a subset. If outputs can be calculated from inputs, then automatically generate the input/output pairs. 
\begin{itemize}
\item To avoid bias in your choice of inputs, use a {\em pseudorandom} input generator, like for example a linear feedback shift register (LFSR);
\item Remember that tests must be {\em reproducible}.
\end{itemize}
\end{block}

\end{frame}

\begin{frame}
\frametitle{Testing in SystemC}
\framesubtitle{Checker}

\begin{block}{The checker reads the outputs}
We usually employ a SystemC method, since we react to MUT outputs with no waits.
\end{block}
\pause
\begin{block}{What if we need to check delays, or in general time dependencies?}
We can always get the current time with \texttt{sc\_time\_stamp()} and store values along with their time for later check.
\begin{itemize}
\item It may be useful to have two processes: one for collection of timed output values ("collector"), and another for final checking;
\item Remember that one test should have a relatively small concern: you should not test multiple time dependencies on the same test.
\end{itemize}
\end{block}
\end{frame}

\begin{frame}
\frametitle{Testing in SystemC}
\framesubtitle{Testbench reusal}

\begin{block}{Repeated code is common when testing}
\begin{itemize}
\item For stimuli, you may have to provide the same input to different modules;
\item For checkers, you may have to verify the same behavior for different input sets.
\end{itemize}
\end{block}
\pause
\begin{block}{Reduce repetition: create stimulus/checker modules}
Now the testbench is simply a container for such modules. 
\begin{enumerate}
\item Introduce an instance of the stimulus and checker modules;
\pause
\item Connect them to the MUT;
\pause
\item Route the checker output to the testbench output;
\pause
\item Simulate and verify the success as before.
\end{enumerate}
\end{block}
\end{frame}