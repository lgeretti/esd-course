\section{Introduction}

\begin{frame}
\frametitle{Introduction}
\framesubtitle{Purpose of the project}

\begin{block}{Learn to design an electronic system}
In particular, create several hardware components and make them interoperate.
\end{block}
\end{frame}

\begin{frame}
\frametitle{Introduction}
\framesubtitle{Approaches that will be used}

\begin{enumerate}
\item Distributed: each one of you can work on different components and collaborate to make your components communicate;
\pause
\item Redundant: you still can develop the same components, if you prefer;
\pause
\item Incremental: you start with simple designs, then continue refining them;
\pause
\item Documented: you create diagrams, use commits, create and manage issues.
\end{enumerate}
\end{frame}

\begin{frame}
\frametitle{Introduction}
\framesubtitle{Expected results}

\begin{block}{The focus is on methodology}
\begin{itemize}
\item Keep your project tidy and use build automation;
\item Use automated tests to verify the correctness of your components;
\item Use fine commits to help yourself and other users exploit your code;
\item Use issue tracking to keep a journal of your development.
\end{itemize}
\end{block}
\end{frame}

\begin{frame}
\frametitle{Introduction}
\framesubtitle{Development and collaboration}

\begin{block}{You are allowed to work from home}
While you can certainly work on the project outside laboratory hours, you don't need to attend lessons: you
have all the tools to asynchronously collaborate at a distance.
\begin{itemize}
\item Nevertheless, it goes to your advantage in terms of communication and support.
\end{itemize}
\end{block}
\pause
\begin{block}{You are allowed to collaborate}
However, your project report is individual, along with the repositories you work on. If you want your contribution to another
repository acknowledged, the corresponding Git commits must have your name.
\begin{itemize} 
\item In the case of shared development, you will need to give write privileges to collaborators in BitBucket.
\end{itemize}
\end{block}

\end{frame}

\begin{frame}
\frametitle{Introduction}
\framesubtitle{Repositories}

\begin{block}{All repositories used for the project must be public}
This is to simplify read access from other users. The important thing is that all repositories can be viewed/cloned by all participants.
\begin{itemize}
\item Also, the repositories used for the project will be evaluated.
\end{itemize}
\end{block}
\pause
\begin{block}{Enable issue tracking for all repositories}
Again, the history of issues will be evaluated.
\end{block}
\pause
\begin{block}{You should not delete/clear your repositories}
It is preferable to keep errors and show your progress.
\begin{itemize}
\item Also, the more issues/commits you can show, the better.
\end{itemize}
\end{block}
\end{frame}

\section{Components}

\begin{frame}
\frametitle{Components}
\framesubtitle{List of components}

\begin{block}{You can develop any and all of such components:}
\begin{itemize}
\item CPU (which has subcomponents)
\item Memory
\item Cache
\end{itemize}
and anything else that is able to communicate with memory (e.g., serial bus).
\end{block}
\end{frame}

\begin{frame}
\frametitle{Components}
\framesubtitle{How to make the CPU work}

\begin{block}{You need an instruction set}
You can design your (simplified) own or pick a subset of your instruction set of choice (e.g., ARM).
\end{block}
\pause
\begin{block}{Suggestion: start simple}
Implement a given set of registers. Then consider only the following operations:
\begin{itemize}
\item Load from memory to chosen register
\item Store to memory from chosen register
\item Arithmetic operations between registers
\end{itemize}
only then you can add jumps, floating points operations, etc.
\end{block}
\end{frame}

\begin{frame}
\frametitle{Components}
\framesubtitle{How to make the memory work}

\begin{block}{Nothing too difficult here}
Allow storing and retrieving data based on address.
\begin{itemize}
\item At some point you may want to introduce multiple read/write ports.
\end{itemize}
\end{block}
\pause
\begin{block}{Properly model delays}
You want to reflect that memory is slower than registers or caches.
\end{block}
\end{frame}

\begin{frame}
\frametitle{Components}
\framesubtitle{How to make the cache work}

\begin{block}{Put it between the load/store operations and the memory}
You can start with a simple policy (store most recently accessed memory blocks, drop the least recent) and then use more sophisticated policies.
\end{block}
\pause
\begin{block}{You can implement different cache levels}
The higher the level, the fastest the simulated retrieval time, but the smaller the available cache size.
\end{block}
\end{frame}

\begin{frame}
\frametitle{Components}
\framesubtitle{How to make an external device work}

\begin{block}{An external device communicates through memory}
The CPU and the device talk by setting values to chosen areas of memory (i.e., "device memory").
\end{block}
\pause
\begin{block}{You need to create the instructions for the device "driver"}
This simply amounts to write a suitable list of instructions that use the device memory.
\end{block}
\end{frame}

\begin{frame}
\frametitle{Components}
\framesubtitle{How to make everything work together}

\begin{block}{Write, run and test a sample program}
\begin{enumerate}
\item Write a list of instructions that statically use a chunk of memory to do something with data, possibly using external devices;
\item Load the instructions into memory;
\item Make sure your program counter points to the right initial instruction;
\item Run the program to completion;
\item Check that the written memory locations have the desired values.
\end{enumerate}
\end{block}
\end{frame}

\section{Report}

\begin{frame}
\frametitle{Report}
\framesubtitle{The basic requirements}

\begin{block}{What to put into your report}
\begin{itemize}
\item Describe the activities performed, with UML diagrams and code snippets where necessary
\item Describe your interaction with other groups
\item Make specific mentions to issues both solved and still opened
\end{itemize}
\end{block}
\end{frame}

\begin{frame}
\frametitle{Report}
\framesubtitle{Where bonus points lie}

\begin{block}{Organization and collaboration are the key}
\begin{itemize}
\item You have clearly organized and coded your components;
\item You have introduced useful and extensive tests;
\item You have used version control and issue tracking in an effective manner;
\item You have employed components from other developers.
\end{itemize}
\end{block}
\end{frame}

\begin{frame}
\frametitle{Report}
\framesubtitle{Submission and evaluation}

\begin{block}{The report will be evaluated before the exam}
Hence you must submit it within 48 hours from the exam you register to.
\begin{itemize}
\item You should submit it by email in electronic format;
\item Earlier submission will allow a more thorough evaluation and consequently it will shorten your oral examination.
\end{itemize}
\end{block}
\pause
\begin{block}{If you do not pass the oral exam, your report is still valid}
You must also keep your repositories available, though. 
In alternative, you can improve your report/repositories and submit the updated report in time for the next exam.
\end{block}
\end{frame}
