\section{Introduction}

\begin{frame}
\frametitle{Introduction}
\framesubtitle{Purpose of the project}

\begin{block}{Learn to design an electronic system}
In particular, create several hardware components and make them interoperate.
\end{block}
\end{frame}

\begin{frame}
\frametitle{Introduction}
\framesubtitle{Approaches that will be used}

\begin{enumerate}
\item Distributed: each one of you can work on different components and collaborate to make your components communicate;
\pause
\item Redundant: you still can develop the same components, if you prefer;
\pause
\item Incremental: you start with simple designs, then continue refining them;
\pause
\item Documented: you create diagrams, use commits, create and manage issues.
\end{enumerate}
\end{frame}

\begin{frame}
\frametitle{Introduction}
\framesubtitle{Remote development}

\begin{block}{You are allowed to work from home}
While you can certainly work on the project outside official laboratory hours, you are not forced to attend lessons: you
now have all the tools to asynchronously collaborate at a distance.
\begin{itemize}
\item Nevertheless, it goes to your advantage in terms of communication and support.
\end{itemize}
\end{block}
\end{frame}

\begin{frame}
\frametitle{Introduction}
\framesubtitle{Expected results}

\begin{block}{The focus is on methodology}
\begin{itemize}
\item Keep your project tidy and use build automation;
\item Use automated tests to verify the correctness of your components;
\item Use fine commits to help yourself and other users exploit your code;
\item Use issue tracking to keep a journal of your development.
\end{itemize}
\end{block}
\end{frame}

\begin{frame}
\frametitle{Introduction}
\framesubtitle{Repositories}

\begin{block}{All repositories used for the project must be public}
This is to simplify access from other users. The important thing is that all repositories can be viewed/cloned by all participants.
\begin{itemize}
\item Also, the repositories used for the project will be evaluated.
\end{itemize}
\end{block}
\pause
\begin{block}{Enable issue tracking for all repositories}
Again, the history of issues will be evaluated.
\end{block}
\pause
\begin{block}{You should not delete your repositories}
It is preferable to keep errors and show your progress.
\begin{itemize}
\item Also, the more (sensible) issues/commits you have made, the better.
\end{itemize}
\end{block}
\end{frame}

\section{Components}



\section{Report}

\begin{frame}
\frametitle{Report}
\framesubtitle{The basic requirements}

\begin{block}{What to put into your report}
\begin{itemize}
\item Describe the activities performed, with UML diagrams and code snippets where necessary
\item Describe your interaction with other groups
\item Make specific mentions to issues both solved and still opened
\end{itemize}
\end{block}
\end{frame}

\begin{frame}
\frametitle{Report}
\framesubtitle{Where bonus points lie}

\begin{block}{Organization and collaboration are the key}
\begin{itemize}
\item You have clearly organized and coded your components;
\item You have introduced useful and extensive tests;
\item You have used version control and issue tracking in an effective manner;
\item You have employed components from other developers.
\end{itemize}
\end{block}
\end{frame}

\begin{frame}
\frametitle{Report}
\framesubtitle{Submission and evaluation}

\begin{block}{The report will be evaluated before the exam}
Hence you must submit it within 48 hours from the exam you register to.
\begin{itemize}
\item Earlier submission will allow a more thorough evaluation and consequently it will ease your oral examination.
\end{itemize}
\end{block}
\pause
\begin{block}{If you do not pass the oral exam, your report is still valid}
You must also keep your repositories available, though. 
In alternative, you can improve your report/repositories and submit the updated report in time for the next exam.
\end{block}
\end{frame}
