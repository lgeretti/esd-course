\section{The program}

\begin{frame}[fragile]
\frametitle{The SystemC program}
\framesubtitle{Structure}

\begin{block}{The body of your executable is:}
\vspace{-1em}
\begin{verbatim}
int sc_main(int argc, char* argv[]) { // Entry
  // Elaboration phase:
  // ...
  sc_start(); // Simulation phase
  // Post-processing phase:
  // ...
  return EXIT_CODE; // Exit (0 means success)
}
\end{verbatim}
\vspace{-1em}
\end{block}
\pause
\begin{block}{Where is the \texttt{main}?}
SystemC hides it within the library, to handle some internal operations.
\end{block}
\end{frame}

\begin{frame}
\frametitle{The SystemC program}
\framesubtitle{The phases}

\begin{block}{Three phases have been identified:}
\begin{enumerate}
\item Elaboration: modules are constructed and connected one another
\item Simulation: processes are executed under the simulator scheduler
\item Post-processing: results from simulation may be evaluated, for example to test behavior and decide success or failure
\end{enumerate}
\end{block}
\pause
This lecture focuses on module declaration, definition and construction.

\end{frame}

\section{Modules}

\subsection{Declaration}

\begin{frame}[fragile]
\frametitle{Modules}
\framesubtitle{Declaration using the SystemC macro}

\begin{block}{A module is a container for state, behavior and structure}
It is comparable to a VHDL entity.
\end{block}
\pause
\begin{block}{The syntax using the \texttt{SC\_MODULE} macro:}
\vspace{-1em}
\begin{verbatim}
#include <systemc.h>
SC_MODULE(module_name) {
  // Module body
};
\end{verbatim}
\vspace{-1em}
\end{block}
\pause
\begin{block}{The body can contain:}
\vspace{-0.5em}
\begin{itemize}
\item Constructors and destructor
\item Ports, channels and data members
\item Other modules
\item Processes
\end{itemize}
\vspace{-0.5em}
\end{block}
\end{frame}

\begin{frame}[fragile]
\frametitle{Modules}
\framesubtitle{Declaration without the SystemC macro}

\begin{block}{What the macro means:}
\vspace{-1em}
\begin{verbatim}
#define SC_MODULE(module_name) \
    struct module_name : public sc_module
\end{verbatim}
\vspace{-1em}
\end{block}
\pause
\begin{block}{However, you should use the following:}
\vspace{-1em}
\begin{verbatim}
class module_name : public sc_module {
  public:
    // Module body
  private:
    // Private parts
};
\end{verbatim}
This is the syntax closest to C++'s "best practice".
\end{block}
\end{frame}

\subsection{Construction}

\begin{frame}[fragile]
\frametitle{Modules}
\framesubtitle{Basic construction}

\begin{block}{A module, being a class, needs a constructor}
Differently from C++, one constructor is required.
\end{block}
\pause
\begin{block}{You can use the \texttt{SC\_CTOR} macro}
\vspace{-1em}
\begin{verbatim}
SC_MODULE(module_name) {
  SC_CTOR(module_name) : <initialization> {
    <submodule allocation>
    <submodule connection>
    <process registration>
    <anything else>
  }
};
\end{verbatim}
\vspace{-1em}
\end{block}

\end{frame}

\begin{frame}[fragile]
\frametitle{Modules}
\framesubtitle{Flexible construction}

\begin{block}{What if we need a different constructor?}
\vspace{0.5em}
{\scriptsize
You need to add the \texttt{SC\_HAS\_PROCESS(module\_name);} macro call.

\begin{itemize}
\item Either within the module declaration:
\begin{verbatim}
SC_MODULE(module_name) {
  SC_HAS_PROCESS(module_name);
  module_name(sc_module_name myname[, other_args...]) 
   : sc_module(myname)[, other_initialization] {
    <the usual constructor body>
  }
};
\end{verbatim}
\pause
\item Or on top of the definition file:
\begin{verbatim}
SC_HAS_PROCESS(module_name);
...
module_name::module_name(sc_module_name myname[, other_args...]) 
   : sc_module(myname)[, other_initialization] {
  <the usual constructor body>
}
\end{verbatim}
\end{itemize}
}
\end{block}

\end{frame}

\subsection{Definition}

\begin{frame}[fragile]
\frametitle{Modules}
\framesubtitle{Definition style: traditional}

\begin{block}{Only the process/helper functions go to the .cpp file}
{\scriptsize
\vspace{-0.5em}
\begin{verbatim}
// name.hpp
#ifndef NAME_HPP
#define NAME_HPP
...
SC_MODULE(Name) {
  ...
  SC_CTOR(Name) : <initializations> {
    <implementation>
  }
  <process declarations>
  <helper declarations>
};
#endif

// name.cpp
#include <systemc.h>
#include "name.hpp"
Name::process_function { <implementation> }
Name::helper_function { <implementation> }
\end{verbatim}
}
\vspace{-0.5em}
\end{block}
\end{frame}

\begin{frame}[fragile]
\frametitle{Modules}
\framesubtitle{Definition style: alternative}

\begin{block}{All definitions go to the .cpp file}
{\scriptsize
\vspace{-0.5em}
\begin{verbatim}
// name.hpp
#ifndef NAME_HPP
#define NAME_HPP
...
SC_MODULE(Name) {
  ...
  SC_CTOR(Name);
  <process declarations>
  <helper declarations>
};
#endif

// name.cpp
#include <systemc.h>
#include "name.hpp"
SC_HAS_PROCESS(Name);
Name::Name(sc_module_name myname) : sc_module(myname), <init> {
  <implementation>
}
...
\end{verbatim}
}
\vspace{-0.8em}
\end{block}
\end{frame}

\subsection{Instantiation}

\begin{frame}[fragile]
\frametitle{Modules}
\framesubtitle{Instantiation}

\begin{block}{You can allocate on the stack or the heap}
{\scriptsize
Stack (preferable):
\vspace{-0.5em}
\begin{verbatim}
#include <systemc.h>
#include "mymodule.hpp"
int sc_main(int argc, char* argv[]) {
  MyModule myModule1("myModule1");
  sc_start();
  return 0;
}
\end{verbatim}
Heap:
\begin{verbatim}
#include <systemc.h>
#include "mymodule.hpp"
int sc_main(int argc, char* argv[]) {
  MyModule* myModule1 = new MyModule("myModule1");
  sc_start();
  delete myModule1;
  return 0;
}
\end{verbatim}
\vspace{-0.8em}
Prefer module names that reflect variable names (useful for debugging).
}
\end{block}
\end{frame}

\section{Submodules}

\subsection{Declaration}

\begin{frame}
\frametitle{Submodules}
\framesubtitle{Declaration}

\begin{block}{Submodules are just module fields of a module class}
No need to introduce additional concepts.
\begin{itemize}
\item Note: Git submodules and CMake subdirectories will be used for any SystemC module {\em external} to the current project
\end{itemize}
\end{block}
\pause
\begin{block}{There are four styles for the definition}
This depends on the use of the stack or heap, and of keeping all the definition within the .cpp file:
\begin{enumerate}
\item Stack, constructors in the .hpp
\item Stack, constructors in the .cpp (preferable)
\item Heap, constructors in the .hpp
\item Heap, constructors in the .cpp
\end{enumerate}
\end{block}
\end{frame}

\subsection{Definition}

\begin{frame}[fragile]
\frametitle{Submodules}
\framesubtitle{Definition: using the stack, constructors in the .hpp}

{\scriptsize
\vspace{-0.5em}
\begin{verbatim}
// car.hpp
#include "body.hpp"
#include "engine.hpp"
SC_MODULE(Car) {
  Body body;
  Engine engine;
  SC_CTOR(Car) : body("body"), engine("engine") {
    // Other initialization
  }
};
\end{verbatim}
}
\end{frame}

\begin{frame}[fragile]
\frametitle{Submodules}
\framesubtitle{Definition: using the stack, constructors in the .cpp}

{\scriptsize
\vspace{-0.5em}
\begin{verbatim}
// car.hpp
#include "body.hpp"
#include "engine.hpp"
SC_MODULE(Car) {
  Body body;
  Engine engine;
  Car(sc_module_name name);
};

// car.cpp
#include <systemc.h>
#include "car.hpp"
SC_HAS_PROCESS(Car);
Car::Car(sc_module_name name) : 
  sc_module(name), body("body"), engine("engine") {
  // Other initialization
}
\end{verbatim}
}
\end{frame}

\begin{frame}[fragile]
\frametitle{Submodules}
\framesubtitle{Definition: using the heap, constructors in the .hpp}

{\scriptsize
\vspace{-0.5em}
\begin{verbatim}
// car.hpp
#include "body.hpp"
#include "engine.hpp"
SC_MODULE(Car) {
  Body* body;
  Engine* engine;
  SC_CTOR(Car) : 
    body(new Body("body")), engine(new Engine("engine")) {
    // Other initialization
  }
};
\end{verbatim}
where the constructor can also be equivalently defined as:
\begin{verbatim}
SC_CTOR(Car) {
  body = new Body("body");
  engine = new Engine("engine");
  // Other initialization
}
\end{verbatim}
}
\end{frame}

\begin{frame}[fragile]
\frametitle{Submodules}
\framesubtitle{Definition: using the heap, constructors in the .cpp}

{\scriptsize
\vspace{-0.5em}
\begin{verbatim}
// car.hpp
class Body;
class Engine;
SC_MODULE(Car) {
  Body* body;
  Engine* engine;
  Car(sc_module_name name);
};

// car.cpp
#include <systemc.h>
#include "car.hpp"
#include "body.hpp"
#include "engine.hpp"
SC_HAS_PROCESS(Car);
Car::Car(sc_module_name name) : 
  sc_module(name), 
  body(new Body("body")), engine(new Engine("engine")) {
  // Other initialization
}
\end{verbatim}
You can use {\em forward class declaration} to make the compilation more efficient.
}
\end{frame}

\subsection{Destruction}

\begin{frame}[fragile]
\frametitle{Submodules}
\framesubtitle{Destruction}

{\scriptsize
\begin{block}{{\small For heap allocation, don't we define the Car destructor?}}
Indeed, we should define the destructor on the containing module (Car), to deallocate the submodules.
\end{block}
\pause
\begin{block}{{\small Destructor declaration and definition:}}
\vspace{-1em}
\begin{verbatim}
// car.hpp
...
SC_MODULE(Car) {
  Body* body;
  Engine* engine;
  Car(sc_module_name name);
  ~Car();
};

// car.cpp
...
Car::~Car() {
  delete body;
  delete engine;
}
\end{verbatim}
\vspace{-1em}
\end{block}
}
\end{frame}

\section{Exercise}

\begin{frame}
\frametitle{Exercise}
\framesubtitle{Objective}

\begin{block}{Create some module classes}
\begin{itemize}
\item Try all the possible definition syntaxes for a module
\item Try using submodules, with different definition and allocation syntaxes
\end{itemize}
\end{block}
\pause
\begin{block}{Topic to be used}
A computer with its different parts (CPU, memory, USB hub, power supply, etc.).
\end{block}
\pause
\begin{block}{Reference repository:}
\url{https://bitbucket.org/uniud_esd/systemc-modules}
\end{block}
\end{frame}
