\section{Introduction}

\begin{frame}
\frametitle{Introduction}
\framesubtitle{How do C and C++ differ?}

\begin{block}{C++ is an {\em object-oriented} language}
\begin{itemize}
\item C++ is fundamentally a superset of C that introduces the concepts of {\em classes} and their {\em objects}
\item C++ allows better organization and maintenance of your code
\end{itemize}
\end{block}

\end{frame}

\begin{frame}
\frametitle{Introduction}
\framesubtitle{Class versus object}


\begin{block}{What is the difference?}
\begin{itemize}
\item A class is a model/template of a entity that has a manipulable state
\item An object is an instance of a class
\end{itemize}
\end{block}
\pause
\begin{block}{An example: Matrix class}
Its model may include:
\begin{itemize}
\item Data {\em fields} that hold the state: e.g., the (multi-dimensional) array of the values
\item {\em Methods} operating on such fields or external arguments: e.g., the method that computes the inverse
\item {\em Operators} : e.g., vector product between two matrices
\end{itemize}
\end{block}

\end{frame}

\begin{frame}
\frametitle{Introduction}
\framesubtitle{What is object-oriented design?}

\begin{block}{It is based on three concepts}
\begin{itemize}
\item Encapsulation: you keep data and functions related to a concept within one class, possibly hiding implementation details; this improves clarity
\item Inheritance: you can create subclasses, inheriting some features of the parent class; this saves development time by reducing duplication
\item Polymorphism: on runtime, you can work on different implementations of a class; this allows to reason in terms of interfaces rather than implementations
\end{itemize}
Our component-based design makes heavy use of polymorphism.
\end{block}

\end{frame}

\begin{frame}
\frametitle{Introduction}
\framesubtitle{Lecture tutorial and references}

\begin{block}{How to follow the steps of this lecture}
Clone the following repository and start from the beginning:
\url{git@bitbucket.org:uniud_esd/cpp_tutorial.git}

\medskip 
\pause
It can be built with CMake, and it does have rudimentary tests to check for failures.
\end{block}
\pause
\begin{block}{C++ references}
Available online:
\begin{itemize}
\item \url{http://www.cplusplus.com/files/tutorial.pdf} \\ (brief tutorial)
\item \url{http://www.cplusplus.com/reference/} \\ (standard library)
\item \url{http://www.mindview.net/Books/DownloadSites} \\ Thinking in C++ (complete book)
\end{itemize}
\end{block}

\end{frame}

\subsection{Encapsulation}

\begin{frame}
\frametitle{Encapsulation}
\framesubtitle{Keep related concepts together}

\begin{block}{Example: an Animal class}
It may have:
\begin{itemize}
\item Fields like race, gender, color, etc.
\item A method for printing how an animal feels towards another animal
\end{itemize}
\end{block}
\pause
\begin{block}{Then we can create Animal objects and use them}
Compare this with having ordered arrays for each field, and an external method that takes two Animal objects and operates on them.
\end{block}
\pause
\begin{block}{C has structs!}
Ok, but no concept of methods or operators, for starters.
\end{block}

\end{frame}

\begin{frame}
\frametitle{Encapsulation}
\framesubtitle{Show only what is necessary}

\begin{block}{Private versus public}
Methods and fields that do not need to be freely accessed should be set private:
\begin{itemize}
\item Fields should be made private when they cannot be set after creation; we can still provide a method to get the value
\item Methods should be made private when they are used internally, and external use may compromise the state of the object
\end{itemize}
\end{block}
\pause
\begin{itemize}
\item C structs are similar to very simple classes with public access and no methods or operators
\end{itemize}

\end{frame}

\subsection{Inheritance}

\begin{frame}
\frametitle{Inheritance}
\framesubtitle{Create classes that extend a base/parent class}

\begin{block}{Subclasses inherit the behavior}
They see everything of the base class if set as public, and also everything set as {\em protected}. Protected fields and methods are private to the rest of the world.
\end{block}
\pause
\begin{block}{Inheritance may be multiple}
This is important if we want a class to implement several interfaces.
\end{block}

\end{frame}

\subsection{Polymorphism}

\begin{frame}
\frametitle{Polymorphism}
\framesubtitle{Change view based on the pointer, change behavior based on the object}

\begin{block}{You can assign an object pointer to a pointer of its parent class}
In this way, you can "see" an object on a more generic level ("upcasting").
\end{block}
\pause
\begin{block}{You can redefine methods by setting them as \texttt{virtual} on the base class}
When using upcasting, you will call the "version" of the method on the actual object.
\end{block}

\end{frame}

\begin{frame}
\frametitle{Polymorphism}
\framesubtitle{Abstract classes and interfaces}

\begin{block}{Probably a "concrete" \texttt{Animal} class is not that useful}
We can create an {\em abstract} class by making at least one method {\em pure virtual}.
\begin{itemize}
\item Pure virtual methods do not need to be defined
\item Objects of abstract classes cannot be created
\end{itemize}
\end{block}
\pause
\begin{block}{An interface is an abstract class with all pure virtual methods}
Nothing of its behavior is defined: only the {\em signatures} of its methods.
\end{block}

\end{frame}

\section{In detail}

\subsection{Class declaration}

\begin{frame}[fragile]
\frametitle{Class declaration}
\framesubtitle{Declarations go to header files (.hpp)}

\begin{block}{Prefer one header file per class, with no definitions}
You name the fields, methods and operators, but you do not provide the definition (i.e., implementation) of methods and operators.
\begin{itemize}
\item In this way, how to interact with the class is separated from how the class behaves
\end{itemize}
\end{block}
\pause
\begin{block}{Minimal example}
Remember to put a semicolon after the class declaration. This is in opposition to definitions, that do not require a semicolon.
\begin{verbatim}
class MyClass { };
\end{verbatim}
\end{block}

\end{frame}

\begin{frame}[fragile]
\frametitle{Class declaration}
\framesubtitle{Field/method/operator visibility}

\begin{block}{Use the minimum access that works}
\begin{itemize}
\item \texttt{private}: (default) is visible only within this class
\item \texttt{protected}: is visible within this class and any subclass
\item \texttt{public}: is visible by anyone 
\end{itemize}
If you want to default as public, use \texttt{struct} instead of \texttt{class}.
\end{block}
\pause
\begin{block}{Visibility is declared in blocks (that can be repeated)}
{ \scriptsize
\texttt{class MyClass \{ } \\
\texttt{  // here we have private visibility until the next access keyword } \\
\texttt{  public: // all public from now on }  \\ 
\texttt{    ... } \\
\texttt{  private: // again all private from now on } \\
\texttt{    ...} \\
\texttt{  protected: // all protected from now on } \\
\texttt{    ...} \\
\texttt{\}; } 
}
\end{block}

\end{frame}

\begin{frame}
\frametitle{Class declaration}
\framesubtitle{Passing by reference or by value}

\begin{block}{When a copy is expensive, better to pass by reference}
Compare the following three method signatures:
\begin{itemize}
\item \texttt{void setDirection(Vector vec); }
\item \texttt{void setDirection(Vector* vec); }
\item \texttt{void setDirection(Vector\& vec); }
\end{itemize}
The first creates a copy of the object within the function, the second creates a copy of the pointer within the function, the third passes the object to the function directly.
\begin{itemize}
\item The second solution is acceptable, but may introduce unnecessary pointers around.
\end{itemize}
\end{block}

\end{frame}

\begin{frame}[fragile]
\frametitle{Class declaration}
\framesubtitle{Passing by reference or by value}

\begin{block}{Passing by reference works also on return values}
\begin{itemize}
\item \texttt{Vector\& invert(Vector\& vec); }
\end{itemize}
Again, this saves creating a copy of the object.
\begin{itemize}
\item Beware of scope: objects on the stack must have scope outside the function! (more on this in the following)
\end{itemize}
\end{block}

\end{frame}

\begin{frame}[fragile]
\frametitle{Class declaration}
\framesubtitle{The \texttt{const} keyword}

\begin{block}{Constness forces immutability}
\begin{itemize}
\item Use \texttt{const} on method/operators to say that they do not modify the state of the object:
\begin{verbatim}
double determinant() const;
\end{verbatim}
\item Use \texttt{const} on arguments and return types to say that they cannot be modified:
\begin{verbatim}
const Vector& getDirection();
void setDirection(const Vector& dir);
\end{verbatim}
\end{itemize}
Using \texttt{const} is not mandatory, but helps you to enforce encapsulation.
\end{block}

\end{frame}

\begin{frame}[fragile]
\frametitle{Class declaration}
\framesubtitle{The \texttt{static} keyword}

\begin{block}{Static means the same for all objects}
You access them with the class name and the double colon:
\begin{verbatim}
MyClass::myField;
MyClass::myMethod();
\end{verbatim}
instead of an object. The field or method must be accessible from outside the class, obviously.
For methods, it also means that the method definition cannot use any non-static field/method of the class. 
\end{block}
\pause
\begin{block}{This is useful for constants or methods that belong to a context}
For example, mathematic methods can be defined as static methods of a Math class, where $\pi$ is a static variable.
\end{block}

\end{frame}

\subsection{Class definition}

\begin{frame}[fragile]
\frametitle{Class definition}
\framesubtitle{Definitions go to source files (.cpp)}

\begin{block}{Header files (.hpp preferably) should hold the class declaration}
You name the fields, methods and operators, but you do not provide the definition (i.e., implementation) of methods and operators.
\begin{itemize}
\item In this way, how to interact with the class is separated from how the class works
\end{itemize}
\end{block}
\pause
\begin{block}{Source files (.cpp preferably) should be created for the definitions}
A class method or operator name is prefixes by the class name and two colons, e.g.

\begin{verbatim}
double Complex::modulus() { ... }
\end{verbatim}
\end{block}

\end{frame}

\begin{frame}[fragile]
\frametitle{Class definition}
\framesubtitle{Constructors}

\begin{block}{A constructor is called to "configure" a new object}
\begin{itemize}
\item It has the name of the class and doesn't have a return type
\item It sets up an object using one or more arguments
\item Multiple constructors can be defined
\item If no constructor is defined, a "default" one with no arguments is created
\end{itemize}
\end{block}
\pause
\begin{block}{You can initialize fields in the constructor definition}
{\scriptsize
\begin{verbatim}
class MyClass {
  int a;
  double b;
public:
  MyClass(int a, double b) : a(a), b(b) { }
};
\end{verbatim}
}
\end{block}
\end{frame}

\begin{frame}[fragile]
\frametitle{Class definition}
\framesubtitle{Destructor}

\begin{block}{A destructor is used to clean up resources}
\begin{itemize}
\item It has the name of the class with a tilde prefix, and doesn't have arguments or a return type
\item Only one destructor can exist
\end{itemize}
\end{block}
\pause
\begin{block}{A destructor needs to be defined if dynamic memory must be freed}
{\scriptsize
\begin{verbatim}
struct MyClass {
  int a;
  double* b;
  MyClass(int a) : a(a), b(new double) { }
  ~MyClass() { delete b; }
};
\end{verbatim}
}
\end{block}

\end{frame}

\begin{frame}[fragile]
\frametitle{Class definition}
\framesubtitle{The \texttt{this} keyword}

\begin{block}{It is a pointer to the current object}
\begin{itemize}
\item It can be used only within method or operator definitions
\item It usually allows to access fields when ambiguity between variables exists:
\begin{verbatim}
MyClass::MyClass(double myField) {
  this->myField = myField;
}
\end{verbatim}
\end{itemize}
It is good practice to use \texttt{this} throughout, since it prevents certain variable scope errors.
\end{block}

\end{frame}

\subsection{Object lifecycle}

\begin{frame}[fragile]
\frametitle{Object lifecycle}
\framesubtitle{Variable allocation}

\begin{block}{The \texttt{new} and \texttt{delete} keywords}
Instead of using \texttt{malloc} and \texttt{free}:

\begin{verbatim}
double* myDouble = new double;
double* myDoubleArray = new double[10];
delete myDouble;
delete[] myDoubleArray;
\end{verbatim}
Compared to C, you do not need to specify the size of the type when allocating.

Allocating objects is similar, but requires further care.
\end{block}

\end{frame}

\begin{frame}
\frametitle{Object lifecycle}
\framesubtitle{Allocation: stack versus heap}

\begin{block}{Compare the following two:}
\begin{enumerate}
\item \texttt{Number num(1.0); }
\item \texttt{Number* num = new Number(1.0); }
\end{enumerate}
Both call the same constructor, but
\begin{enumerate}
\item The object is allocated in the {\em stack}, hence the object is destroyed when the variable loses scope
\item The object is allocated in the {\em heap} (and the pointer in the stack), hence the object is not destroyed
\end{enumerate}
Similarly to C pointers, we can "pass" an object in the heap by copying pointers to the object.
\end{block}

\end{frame}

\begin{frame}
\frametitle{Object lifecycle}
\framesubtitle{Allocating an array}

\begin{block}{Same as before:}
\begin{enumerate}
\item {\scriptsize \texttt{Number nums[2]; }}
\item {\scriptsize \texttt{Number* nums = new Number[2]; }}
\end{enumerate}
where you must consider that constructors are still called: {\em an object variable is always initialized}. 

If no constructor with zero arguments exists, this does not even compile. 
\end{block}
\begin{block}{You can initialize the array using any constructor:}
\begin{enumerate}
\item {\scriptsize \texttt{Number nums[2] \{ Number(1.0), Number(2.0) \}; } }
\item {\scriptsize \texttt{Number* num = new Number[2] \{ Number(1.0), Number(2.0) \}; } }
\end{enumerate}
If you instead need to assign the objects later, use an array of {\em pointers} to objects.
\end{block}

\end{frame}

\begin{frame}
\frametitle{Object lifecycle}
\framesubtitle{Cleaning up}

\begin{block}{Stack allocation should be preferred}
The destructor is called automatically, so there is little ground for errors. We only need to be clear about the lifetime of the object.
\end{block}
\pause
\begin{block}{When heap allocation is necessary, clean up}
\begin{itemize}
\item \texttt{delete variableName;} \,\,\, if it is a single variable
\item \texttt{delete[] variableName;} \,\,\, if it is an array
\end{itemize}
\end{block}
\end{frame}

\subsection{Inheritance}

\begin{frame}
\frametitle{Inheritance}
\framesubtitle{Derive from a base class}

\begin{block}{You can derive as public or private/protected}
Defaults to private for a class, public for a struct. You need to use public to create a "is-a" relationship.

{\scriptsize 
\texttt{class Derived : public Base \{ ... \}; }
}
\end{block}
\pause
\begin{block}{You can derive from multiple classes}
The set of "functionality" you have is the union of the functionalities from each parent class.

{\scriptsize 
\texttt{class Derived2 : public BaseA, public BaseB \{ ... \}; }
}
\end{block}
\pause
\begin{block}{When defining a constructor, you can call the parent constructor}
Remember that constructors are not inherited.

\medskip

{\scriptsize 
\texttt{Derived::Derived(int a) : Base(a) \{ ... \} } \\
\texttt{Derived2::Derived2(int x, int y) : BaseA(x), BaseB(y) \{ ... \} }
}
\end{block}
\end{frame}

\subsection{Polymorphism}

\begin{frame}[fragile]
\frametitle{Polymorphism}
\framesubtitle{Virtual methods}

\begin{block}{By marking with \texttt{virtual}, you allow subclasses to {\em override} the method}
This is important if a subclass can be more specific.

{\scriptsize 
\texttt{virtual void printDescriptionMessage() const; }
}
\end{block}
\pause
\begin{block}{You can assign an object to a variable related to one of its superclasses ("upcast")}
{\scriptsize 
\texttt{Base* base = new Base(); } \\
\texttt{Derived* derived = new Derived(); } \\
\texttt{Base* upcastToBase = derived; } \\
\texttt{base.printDescriptionMessage(); } \\
\texttt{derived.printDescriptionMessage(); } \\
\texttt{upcastToBase.printDescriptionMessage(); } \\
}
The last two call the overridden version, i.e., the one of the actual object. We say that 
virtual methods operate on the {\em dynamic type}, not the {\em static type} given by the variable type.
\end{block}

\end{frame}

\begin{frame}[fragile]
\frametitle{Polymorphism}
\framesubtitle{Calling overridden methods}

\begin{block}{How to call the base method within the derived class?}
Just use the name of the base class with the two colons:
{\scriptsize 
\begin{verbatim}
class Derived : public Base {
  ...
  
  void printDescriptionMessage() {
    Base::printDescriptionMessage();
    ...
  }
};
\end{verbatim}
}
Happens sometimes, especially when we want to {\em extend} rather than {\em substitute} the behavior of a method.
\end{block}
\pause
\begin{block}{Note how we did not mark the overridden method as virtual itself}
We do it only if we want to allow classes derived from \texttt{Derived} to override it.
\end{block}

\end{frame}

\begin{frame}[fragile]
\frametitle{Polymorphism}
\framesubtitle{Pure virtual methods}

\begin{block}{What if you do not even want to define the method in the base class?}
Use a pure virtual function:

{\scriptsize 
\texttt{virtual int getNumSides() const = 0; }
}
\end{block}
\pause
\begin{block}{A pure virtual method makes a class abstract}
You cannot create an object (or an array of objects) for it, but you can still assign a pointer to an object of a subclass:

{\scriptsize 
\begin{verbatim}
string dogString = "dog";
Dog* myDog = new Dog(dogString,4);
Animal* myAnimal;
myAnimal = myDog;
\end{verbatim}
}

\end{block}

\end{frame}

\begin{frame}[fragile]
\frametitle{Polymorphism}
\framesubtitle{An example of abstracted behavior}
{\scriptsize 
\begin{verbatim}
struct Shape {
  virtual int getNumSides() const = 0;
};

struct Triangle : Shape {
  virtual int getNumSides() const { return 3; }
};

struct Square : Shape {
  virtual int getNumSides() const { return 4; }
};
\end{verbatim}
}
With:
{\scriptsize 
\begin{verbatim}
Shape** shapes = new Shape*[2];
shapes[0] = new Triangle();
shapes[1] = new Square();
  
int sides[2];
for (int i=0;i<2;i++)
  sides[i] = shapes[i]->getNumSides();
\end{verbatim}
}
\end{frame}

\begin{frame}[fragile]
\frametitle{Polymorphism}
\framesubtitle{Pure virtual methods still {\em can} be defined}

\begin{block}{If a definition exists, then what's the difference with a non-pure virtual function?}
Let's consider three types of methods in the base class (B: base class, D: derived class):
\begin{enumerate}
\item Virtual: B version is callable within D implementation, D {\em may not override} the method, and B may be non-abstract (i.e., B version could also be called on B objects);
\pause
\item Pure virtual: B version is not callable within D implementation, D {\em must implement} the method, and B is abstract;
\pause
\item Pure virtual and defined: B version is callable within D implementation, D {\em must override} the method, and B is abstract;
\end{enumerate}
\end{block}

\end{frame}

\begin{frame}[fragile]
\frametitle{Polymorphism}
\framesubtitle{Dynamic casting}

\begin{block}{What if we want to "downcast"?}
Downcasting means to recover a specific object from an abstracted pointer.
We can use {\em dynamic casting}:

{\scriptsize 
\begin{verbatim}
// Given a base_ptr of type Base*
Derived1* der1_ptr = dynamic_cast<Derived1*>(base_ptr);
Derived2* der1_ptr = dynamic_cast<Derived2*>(base_ptr);
\end{verbatim}
}
This compiles if the casting is legal, i.e., the classes are derived from \texttt{Base}.
\end{block}
\begin{block}{Which cast succeeds depends on the dynamic type}
By checking which pointer is not null (or equivalently, false), you get which dynamic type is the correct one and work with the right pointer.
\end{block}

\end{frame}

\begin{frame}
\frametitle{Polymorphism}
\framesubtitle{Interfaces versus abstract classes}

\begin{block}{What is the difference?}
\begin{itemize}
\item Interfaces only have pure virtual undefined methods, with no fields
\item Abstract classes have at least one pure virtual method
\end{itemize}
\end{block}
\pause
\begin{block}{When to use each?}
\begin{itemize}
\item Use interfaces wherever possible, especially if classes are exposed to users (i.e., if they are not "internal")
\item Use abstract classes between interfaces and classes, if you have multiple derived classes with common fields/methods
\end{itemize}
This approach improves clarity for users of your library, and shields you from breaking the interfacing when you change some internal implementation detail.
\end{block}

\end{frame}

\subsection{Extras}

\begin{frame}[fragile]
\frametitle{Extras}
\framesubtitle{Template programming}

\begin{block}{Templating provides methods that apply to multiple class}
We do not explain how to write templates: it is only relevant because SystemC uses templated classes.
\end{block}
\pause
\begin{block}{Example: templated class usage}
\begin{verbatim}
std::vector<MyClass*> vec;
vec.push_back(new MyClass());
\end{verbatim}

Here adding items to a vector is the same whatever class is used, so the \texttt{std::vector} class is written with no regards to the objects it manipulates.
\end{block}

\end{frame}

\begin{frame}[fragile]
\frametitle{Extras}
\framesubtitle{Enumerations}

\begin{block}{Enumerations are type-safe sets of values}
Instead of using arbitrary strings or numbers, you can write down the exact possible values:

{\scriptsize 
\begin{verbatim}
enum class Feeling { HATRED, INDIFFERENCE, FEAR, CURIOSITY};
\end{verbatim}
}
\end{block}
\pause
\begin{block}{To create a variable and compare it:}

{\scriptsize 
\begin{verbatim}
Feeling feeling = Feeling::CURIOSITY;
if (age != Feeling::INDIFFERENCE)
  doSomething();
\end{verbatim}
}

This is a safer and more efficient approach. Also, enumerations can be used in \texttt{switch} statements.
\end{block}

\end{frame}

\begin{frame}[fragile]
\frametitle{Extras}
\framesubtitle{Standard output}

\begin{block}{Instead of function calls, {\em streams} are used}

{\scriptsize 
\begin{verbatim}
string a = "test";
int b = 1;
std::cout << "This is " << b << " " << a << std::endl;
\end{verbatim}
}

Essentially, you concatenate text into the standard output. It is not necessary to format variables.
\end{block}
\pause
\begin{block}{You could overload the \texttt{<\;\!\!<} operator to handle a specific class}
{\scriptsize 
\begin{verbatim}
std::ostream& operator<<(std::ostream& os, const MyClass& obj) {
    return os << obj.methodThatReturnsSomeText();
}
\end{verbatim}
}
\end{block}

\end{frame}

