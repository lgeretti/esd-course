\section{Introduction}

\begin{frame}
\frametitle{Introduction}
\framesubtitle{How do C and C++ differ?}

\begin{block}{C++ is an {\em object-oriented} language}
\begin{itemize}
\item C++ is fundamentally a superset of C that introduces the concepts of {\em classes} and their {\em objects}
\item C++ allows better organization and maintenance of your code
\end{itemize}
\end{block}

\end{frame}

\begin{frame}
\frametitle{Introduction}
\framesubtitle{Class versus object}


\begin{block}{What is the difference?}
\begin{itemize}
\item A class is a model/template of a entity that has a manipulable state
\item An object is an instance of a class
\end{itemize}
\end{block}
\pause
\begin{block}{An example: Matrix class}
Its model may include:
\begin{itemize}
\item Data {\em fields} that hold the state: e.g., the (multi-dimensional) array of the values
\item {\em Methods} operating on such fields or external arguments: e.g., the method that computes the inverse
\item {\em Operators} : e.g., vector product between two matrices
\end{itemize}
\end{block}

\end{frame}

\begin{frame}
\frametitle{Introduction}
\framesubtitle{What is object-oriented design?}

\begin{block}{It is based on three concepts}
\begin{itemize}
\item Encapsulation: you keep data and functions related to a concept within one class, possibly hiding implementation details; this improves clarity
\item Inheritance: you can create subclasses, inheriting some features of the parent class; this saves development time by reducing duplication
\item Polymorphism: on runtime, you can pass different implementations of an object; this allows to reason in terms of interfaces rather than implementations
\end{itemize}
Our component-based design makes heavy use of polymorphism.
\end{block}

\end{frame}

\begin{frame}
\frametitle{Introduction}
\framesubtitle{Lecture tutorial and references}

\begin{block}{How to follow the steps of this lecture}
Clone the following repository and start from the beginning:
\url{git@bitbucket.org:uniud_esd/cpp_tutorial.git}
\medskip 

It can be built with CMake, and it does have rudimentary tests to check for failures.
\end{block}
\pause
\begin{block}{C++ references}
Available online:
\begin{itemize}
\item \url{http://www.cplusplus.com/files/tutorial.pdf} \\ (brief tutorial)
\item \url{http://www.mindview.net/Books/DownloadSites} \\ Thinking in C++ (complete book)
\end{itemize}
\end{block}

\end{frame}

\subsection{Encapsulation}

\begin{frame}
\frametitle{Encapsulation}
\framesubtitle{Keep related concepts together}

\begin{block}{Example: the Animal class}
It may have:
\begin{itemize}
\item Fields like race, gender, color, etc.
\item A method for printing how an animal feels towards another animal
\end{itemize}
\end{block}
\pause
\begin{block}{Then we can create Animal objects and use them}
Compare this with having ordered arrays for each field, and an external method that takes two Animal objects and returns the result.
\end{block}

\end{frame}

\begin{frame}
\frametitle{Encapsulation}
\framesubtitle{Show only what is necessary}

\begin{block}{Private versus public}
Methods and fields that do not need to be freely accessed should be set private:
\begin{itemize}
\item Fields should be made private when they cannot be set after creation; we can still provide a method to get the value
\item Methods should be made private when they are used internally, and external use may compromise the state of the object
\end{itemize}
\end{block}
\pause
\begin{itemize}
\item C structs are similar to very simple classes with public access and no methods or operators
\end{itemize}

\end{frame}

\subsection{Inheritance}

\begin{frame}
\frametitle{Inheritance}
\framesubtitle{Create classes that extend a base/parent class}

\begin{block}{Subclasses inherit the behavior}
They see everything of the base class if set as public, and also everything set as {\em protected}. Protected fields and methods are private to the rest of the world.
\end{block}
\pause
\begin{block}{Subclasses can override a method, if set as {\em virtual} on the base class}
In this way, you can provide a generic implementation and rewrite it for specific cases.
\end{block}
\end{frame}

\subsection{Polymorphism}

\begin{frame}
\frametitle{Polymorphism}
\framesubtitle{Deal with superclass variables}

\begin{block}{How do we exploit the \texttt{Dog} and \texttt{Cat} classes in the test?}
The trick: we can assign an object of a class to a variable of its parent class.
\end{block}
\pause
\begin{block}{The implementation of \texttt{feelsAbout} is decided at runtime based on the actual object on which it is called}
In this way, you can ignore the details of the implementation and focus on the available methods.
\begin{itemize}
\item You can also check if a superclass variable holds an object of a specific class
\end{itemize}
\end{block}

\end{frame}

\begin{frame}
\frametitle{Polymorphism}
\framesubtitle{Abstract classes and interfaces}

\begin{block}{Probably a "concrete" \texttt{Animal} class is not that useful}
We can create an {\em abstract} class by making at least one method {\em pure virtual}.
\begin{itemize}
\item Pure virtual methods do not need to be defined
\item Objects of abstract classes cannot be created
\end{itemize}
\end{block}
\pause
\begin{block}{An interfaces is an abstract class with all pure virtual methods}
Nothing of its behavior is defined: only the {\em signatures} of its methods.
\end{block}

\end{frame}

\section{In detail}

\subsection{Class declaration}

\begin{frame}[fragile]
\frametitle{Class declaration}
\framesubtitle{Declarations go to header files (.hpp)}

\begin{block}{Prefer one header file per class, with no definitions}
You name the fields, methods and operators, but you do not provide the definition (i.e., implementation) of methods and operators.
\begin{itemize}
\item In this way, how to interact with the class is separated from how the class behaves
\end{itemize}
\end{block}
\pause
\begin{block}{Minimal example}
Remember to put a semicolon after the class declaration. This is in opposition to definitions, that do not require a semicolon.
\begin{verbatim}
class MyClass { };
\end{verbatim}
\end{block}

\end{frame}

\begin{frame}[fragile]
\frametitle{Class declaration}
\framesubtitle{Field/method/operator visibility}

\begin{block}{Use the minimum access that works}
\begin{itemize}
\item \texttt{private}: (default) is visible only within this class
\item \texttt{protected}: is visible within this class and any subclass
\item \texttt{public}: is visible by anyone 
\end{itemize}
If you want to default as public, use \texttt{struct} instead of \texttt{class}.
\end{block}
\pause
\begin{block}{Visibility is declared in blocks (that can be repeated)}
{ \scriptsize
\texttt{class MyClass \{ } \\
\texttt{  // private visibility here } \\
\texttt{  public: } \\
\texttt{    ... } \\
\texttt{  private: // again } \\
\texttt{    ...} \\
\texttt{  protected:} \\
\texttt{    ...} \\
\texttt{\}; } 
}
\end{block}

\end{frame}

\begin{frame}
\frametitle{Class declaration}
\framesubtitle{Passing by reference or by value}

\begin{block}{To pass by reference is preferable when copying is expensive}
Compare the following three:
\begin{itemize}
\item \texttt{void setDirection(Vector vec); }
\item \texttt{void setDirection(Vector* vec); }
\item \texttt{void setDirection(Vector\& vec); }
\end{itemize}
The first creates a copy of the object within the function, the second creates a copy of the pointer within the function, the third passes the object to the function directly.
\begin{itemize}
\item The second solution is acceptable, but may introduce unnecessary pointers around.
\end{itemize}
\end{block}

\end{frame}

\begin{frame}[fragile]
\frametitle{Class declaration}
\framesubtitle{Passing by reference or by value}

\begin{block}{Passing by reference works also on return values}
\begin{itemize}
\item \texttt{Vector\& invert(Vector\& vec); }
\end{itemize}
Again, this saves creating a copy of the object.
\begin{itemize}
\item Beware of scope: objects on the stack must have scope outside the function! (more on this in the following)
\end{itemize}
\end{block}

\end{frame}

\begin{frame}[fragile]
\frametitle{Class declaration}
\framesubtitle{The \texttt{const} keyword}

\begin{block}{Constness forces immutability}
\begin{itemize}
\item Use \texttt{const} on method/operators to say that they do not modify the state of the object:
\begin{verbatim}
double determinant() const;
\end{verbatim}
\item Use \texttt{const} on arguments and return types to say that they cannot be modified:
\begin{verbatim}
const Vector& getDirection();
void setDirection(const Vector& dir);
\end{verbatim}
\end{itemize}
Using \texttt{const} is not mandatory, but helps you to enforce encapsulation.
\end{block}

\end{frame}

\subsection{Class definition}

\begin{frame}[fragile]
\frametitle{Class definition}
\framesubtitle{Definitions go to source files (.cpp)}

\begin{block}{Header files (.hpp preferably) should hold the class declaration}
You name the fields, methods and operators, but you do not provide the definition (i.e., implementation) of methods and operators.
\begin{itemize}
\item In this way, how to interact with the class is separated from how the class works
\end{itemize}
\end{block}
\pause
\begin{block}{Source files (.cpp preferably) should be created for the definitions}
A class method or operator name is prefixes by the class name and two colons, e.g.

\begin{verbatim}
double Complex::modulus() { ... }
\end{verbatim}
\end{block}

\end{frame}

\begin{frame}
\frametitle{Class definition}
\framesubtitle{Constructors and destructor}

\begin{block}{A constructor is called to "configure" a new object}
\begin{itemize}
\item It has the name of the class and doesn't have a return type
\item It sets up an object using one or more arguments
\item If no constructor is defined, a "default" one is created
\item Multiple constructors can be defined
\end{itemize}
\end{block}
\pause
\begin{block}{A destructor is used to clean up resources}
\begin{itemize}
\item It has the name of the class with a tilde prefix, and doesn't have arguments or a return type
\item Only one destructor can exist
\item It needs to be explicitly defined only if dynamic memory is allocated for the object fields
\end{itemize}
\end{block}

\end{frame}

\begin{frame}[fragile]
\frametitle{Class definition}
\framesubtitle{The \texttt{this} keyword}

\begin{block}{It is a pointer to the current object}
\begin{itemize}
\item It can be used only within method or operator definitions
\item It usually allows to access fields when ambiguity between variables exists:
\begin{verbatim}
MyClass::MyClass(double myField) {
  this->myField = myField;
}
\end{verbatim}
\end{itemize}
It is good practice to use \texttt{this} throughout, since it prevents certain variable scope errors.
\end{block}

\end{frame}

\subsection{Object lifecycle}

\begin{frame}[fragile]
\frametitle{Object lifecycle}
\framesubtitle{Variable allocation}

\begin{block}{The \texttt{new} and \texttt{delete} keywords}
Instead of using \texttt{malloc} and \texttt{free}:

\begin{verbatim}
double* myDouble = new double;
double* myDoubleArray = new double[10];
delete myDouble;
delete[] myDoubleArray;
\end{verbatim}
Compared to C, you do not need to specify the size of the type when allocating.

Allocating objects is similar, but requires further care.
\end{block}

\end{frame}

\begin{frame}
\frametitle{Object lifecycle}
\framesubtitle{Allocation: stack versus heap}

\begin{block}{Compare the following:}
\begin{itemize}
\item \texttt{Number num(1.0); }
\item \texttt{Number* num = new Number(1.0); }
\end{itemize}
Both call the same constructor, but
\begin{itemize}
\item The first object is allocated in the {\em stack}, hence the object is destroyed when the variable loses scope
\item The second object is allocated in the {\em heap} (and the pointer in the stack), hence the object is not destroyed
\end{itemize}
Similarly to C pointers, we can "pass" an object in the heap by copying pointers to the object.
\end{block}

\end{frame}

\begin{frame}
\frametitle{Object lifecycle}
\framesubtitle{Allocating an array}

\begin{block}{Same as before:}
\begin{itemize}
\item \texttt{Number nums[2]; }
\item \texttt{Number* nums = new Number[2]; }
\end{itemize}
where you must consider that constructors are still called. If no constructor with zero arguments exists, this does not even compile. In C++11, you can initialize with
any constructor by using:
\begin{itemize}
\item \texttt{Number nums[2] \{ Number(1.0), Number(2.0) \}; }
\item \texttt{Number* num = new Number[2] \{ Number(1.0), Number(2.0) \}; }
\end{itemize}
\end{block}

\end{frame}

\begin{frame}
\frametitle{Object lifecycle}
\framesubtitle{Cleaning up}

\begin{block}{Stack allocation should be preferred}
The destructor is called automatically, so there is little ground for errors. We only need to be clear about the lifetime of the object.
\end{block}
\pause
\begin{block}{When heap allocation is necessary, clean up}
\begin{itemize}
\item \texttt{delete variableName;} \,\,\, if it is a single variable
\item \texttt{delete[] variableName;} \,\,\, if it is an array
\end{itemize}
\end{block}
\end{frame}

\subsection{Inheritance}

\begin{frame}
\frametitle{Inheritance}
\framesubtitle{Derive from a base class}

\begin{block}{You can derive as public or private/protected}
Defaults to private for a class, public for a struct. You use public if you want to create a "is-a" relationship.

{\scriptsize 
\texttt{class Derived : public Base \{ ... \}; }
}
\end{block}
\pause
\begin{block}{You can derive from multiple classes}
The set of "functionality" you have is the union of the functionalities from each parent class.

{\scriptsize 
\texttt{class Derived2 : public BaseA, public BaseB \{ ... \}; }
}
\end{block}
\pause
\begin{block}{When defining a constructor, you can call the parent constructor}
Remember that constructors are not inherited.

\medskip

{\scriptsize 
\texttt{Derived::Derived(int a) : Base(a) \{ ... \} } \\
\texttt{Derived2::Derived2(int x, int y) : BaseA(x), BaseB(y) \{ ... \} }
}
\end{block}
\end{frame}

\begin{frame}[fragile]
\frametitle{Inheritance}
\framesubtitle{Virtual methods}

\begin{block}{By marking with \texttt{virtual}, you allow subclasses to rewrite the method}
This is important if a subclass can be more specific.

{\scriptsize 
\texttt{virtual string getDescriptionMessage() const; }
}
\end{block}
\pause
\begin{block}{How to call the method on the base class?}
Just use the name of the base class with the two colons:
{\scriptsize 
\begin{verbatim}
class Derived : public Base {
  ...
  
  void doSomething() {
    Base::doSomething();
    ...
  }
};
\end{verbatim}
}
Happens sometimes, especially when we want to {\em extend} rather than {\em substitute} the behavior of a method.
\end{block}
\end{frame}

\begin{frame}[fragile]
\frametitle{Inheritance}
\framesubtitle{Pure virtual methods}

\begin{block}{What if you do not want to define the method in the base class?}
Use a pure virtual function:

{\scriptsize 
\texttt{virtual int getNumSides() const = 0; }
}
\end{block}
\pause
\begin{block}{A pure virtual method makes a class abstract}
You cannot create objects for it, but you can assign objects of any of its subclasses to a pointer variable of its type:

{\scriptsize 
\begin{verbatim}
string dogString = "dog";
Dog* myDog = new Dog(dogString,4);
Animal* myAnimal;
myAnimal = myDog;
\end{verbatim}
}
The only "drawback" is that you can now see only the functionality of an \texttt{Animal}.
\end{block}

\end{frame}


\begin{frame}[fragile]
\frametitle{Inheritance}
\framesubtitle{An example of abstraction}
{\scriptsize 
\begin{verbatim}
struct Shape {
  virtual int getNumSides() const = 0;
};

struct Triangle : Shape {
  virtual int getNumSides() const { return 3; }
};

struct Square : Shape {
  virtual int getNumSides() const { return 4; }
};
\end{verbatim}
}
And:
{\scriptsize 
\begin{verbatim}
std::vector<Shape*> shapes;
shapes.push_back(new Triangle());
shapes.push_back(new Square());

int sides[shapes.size()];  
for (int i=0; i<shapes.size(); i++)
  sides[i] = shapes[i]->getNumSides();
\end{verbatim}
}
\end{frame}