\section{Introduction}

\begin{frame}
\frametitle{Introduction}
\framesubtitle{Contents}
\begin{block}{We will cover dynamism by discussing about:}
\begin{enumerate}
\item Time: how SystemC explicitly deals with simulated time;
\item Processes: how modules can do things
\begin{itemize}
\item Threads
\item Methods
\end{itemize}
\item Events: how modules can synchronize between each other.
\end{enumerate}
\end{block}
\pause
\begin{block}{Repository for examples:}
\url{https://bitbucket.org/uniud_esd/systemc-processes}
\end{block}
\end{frame}

\begin{frame}
\frametitle{Introduction}
\framesubtitle{Simulation evolution}

\begin{block}{How does simulation work?}
It is a succession of {\em evaluation} and {\em advancement} phases:
\begin{enumerate}
\item Evaluation: each runnable process executes until a \texttt{return} or a \texttt{wait} is issued;
\begin{itemize}
\item A \texttt{wait} for a process removes that process from the runnable set, and puts an {\em event} into an {\em event queue}: the event has an associated time given by the final time of the wait period;
\item A \texttt{return} for a process clearly concludes the simulation for that process.
\end{itemize}
Please note how the order in which the processes run during the evaluation phase is irrelevant.
\pause
\item Advancement: simulated time is advanced up to the next event in the event queue, where some processes may again be put into the runnable state.
\end{enumerate}
\end{block}

\end{frame}

\section{Time}

\begin{frame}
\frametitle{Time}
\framesubtitle{Real and simulated times are different}

\begin{block}{We identify three kinds of times:}
\begin{enumerate}
\item Wall-clock time: the real time between start and end of an operation, including wait times;
\pause
\item Processor time: the real time actually spent computing during such an operation, thus excluding the idle time;
\pause
\item Simulated time: the time between start and end of an operation expressed under a time model for the simulated system.
\end{enumerate}
\end{block}
\end{frame}

\begin{frame}[fragile]
\frametitle{Time}
\framesubtitle{Simulated time}

\begin{block}{SystemC uses the \texttt{sc\_time} type}
\vspace{0.5em}
It has at least 64 bit precision, and can be specified with a time unit:
\vspace{-0.5em}
{\scriptsize 
\begin{verbatim}
sc_time name; // no initialization
sc_time name(double, sc_time_unit); // set value and unit
\end{verbatim}
}
\vspace{-0.5em}
\end{block}
\pause
\begin{block}{Units}
\begin{table}
\begin{tabular}{lll}
\hline
enum & Units & Magnitude \\
\hline
\texttt{SC\_FS} & femtoseconds & {\small $10^{-15}$} \\
\texttt{SC\_PS} & picoseconds & {\small $10^{-12}$} \\
\texttt{SC\_NS} & nanoseconds & {\small $10^{-9}$} \\
\texttt{SC\_US} & microseconds & {\small $10^{-6}$} \\
\texttt{SC\_MS} & milliseconds & {\small $10^{-3}$} \\
\texttt{SC\_SEC} & seconds & {\small $10^{-0}$} \\
\end{tabular}
\end{table}
\vspace{-1em}
\end{block}
\end{frame}

\begin{frame}
\frametitle{Time}
\framesubtitle{Resolution}

\begin{block}{All \texttt{sc\_time} objects have a common resolution}
This can be chosen with
\begin{itemize}
\item \texttt{sc\_set\_time\_resolution(double, sc\_time\_unit)}
\end{itemize}
Where the first parameter is a non-negative power of ten (e.g. 1, 10, 100, etc.).
\end{block}
\pause
\begin{block}{Setting the resolution must be done at the beginning}
\begin{itemize}
\item No \texttt{sc\_time} objects must have been created yet;
\item Simulation must not have started yet.
\end{itemize}
\end{block}
\end{frame}

\begin{frame}[fragile]
\frametitle{Time}
\framesubtitle{Dealing with time objects}

\begin{block}{Operating with other time objects or numeric values}
\begin{itemize}
\item Comparison: \, \verb@== != < <= > >=@
\item Arithmetic: \, \verb@+ - * /@
\item Assignment: \, \verb@= += -= *= /=@
\end{itemize}
Also, the \verb@operator<<@ is provided, i.e., you can print time objects.
\end{block}
\pause
\begin{block}{Getting the current simulation time}
You can use \verb+sc\_time\_stamp()+: \\
\begin{verbatim}
sc_time current_time = sc_time_stamp();
\end{verbatim}
This is done within code that runs during the simulation.
\end{block}
\end{frame}

\begin{frame}[fragile]
\frametitle{Time}
\framesubtitle{Starting time evolution}

\begin{block}{You call \texttt{sc\_start()} globally}
\vspace{0.5em}
Simulation runs until no more processing is due, or up to a provided maximum time:
\begin{itemize}
\item No limit: {\small \verb@sc_start();@ }
\item Absolute limit: {\small \verb@sc_start(const sc_time& max_sc_time);@ }
\item Relative limit: {\small \verb@sc_start(double max_time, sc_time_unit time_unit);@ }
\end{itemize}
In any case, you can start the simulation {\em once and once only} for one call of \texttt{sc\_main}, i.e., once per executable.
\end{block}
\pause
\begin{block}{To get the simulated time, call \texttt{sc\_time\_stamp()}}
You will do it within any process that is run by the simulator, or after the end of simulation.
\end{block}
\end{frame}

\begin{frame}[fragile]
\frametitle{Time}
\framesubtitle{Pausing time evolution}

\begin{block}{You can call \texttt{wait(sc\_time)} on any \texttt{SC\_THREAD}}
\vspace{0.5em}
You have two ways of calling the wait function:
{\scriptsize 
\begin{verbatim}
void MyModule::myProcess(void) {
  wait(10, SC_NS);
  cout << "Now at " << sc_time_stamp() << endl;
  sc_time delay(2, SC_MS);
  wait(delay);
  cout << "Now at " << sc_time_stamp() << endl;
}
\end{verbatim}
}
\pause
The above code will return the following (assuming a starting time of 0 ns):
{\scriptsize 
\begin{verbatim}
Now at 10 ns
Now at 4000010 ns
\end{verbatim}
}
\end{block}
\end{frame}

\section{Threads}

\begin{frame}
\frametitle{Threads}
\framesubtitle{Registering a thread}

\begin{block}{You register with \texttt{SC\_THREAD(method\_name);}}
A couple of restrictions:
\begin{enumerate}
\item The \texttt{SC\_THREAD} call must be performed during the elaboration stage (for example, within the constructor);
\item The method must exist for the module, it must return \texttt{void} and have no arguments.
\end{enumerate}
\end{block}
\pause
\begin{block}{A thread can still run indefinitely}
You only need to create an infinite loop inside.
\begin{itemize}
\item To return control to the scheduler, you will use \texttt{wait} calls.
\end{itemize}
\end{block}
\end{frame}

\begin{frame}[fragile]
\frametitle{Threads}
\framesubtitle{Example of two concurrent thread processes}

{\scriptsize 
\begin{verbatim}
// twoprocesses.hpp
SC_MODULE(TwoProcesses) {
  void wiper_thread(void);
  void blinker_thread(void);
  SC_CTOR(TwoProcesses) {
    SC_THREAD(wiper_thread);
    SC_THREAD(blinker_thread);
  }
};
// twoprocesses.cpp
void TwoProcesses::wiper_thread(void) {
  while (true) {
    wipe_left();
    wait(3,SC_SEC);
    wipe_right();
    wait(3,SC_SEC);
}}
void TwoProcesses::wiper_thread(void) {
  while (true) {
    blink_on();
    wait(2,SC_SEC);
    blink_off();
    wait(2,SC_SEC);
}}
\end{verbatim}
}

\end{frame}

\section{Events}

\begin{frame}
\frametitle{Events}
\framesubtitle{What is an event?}

\begin{block}{Definition}
A SystemC event is something that occurs at a specific instant in time. It has no value and no duration.
\begin{itemize}
\item We have already seen an implicit event: the timeout for a \texttt{wait(sc\_time)} call.
\end{itemize}
\end{block}
\pause
\begin{block}{Explicit events: the \texttt{sc\_event} type}
It allows to notify (cause) an event, for other processes to wait for.
\end{block}

\end{frame}

\begin{frame}[fragile]
\frametitle{Events}
\framesubtitle{Causing an event}

\begin{block}{Syntaxes:}
\begin{itemize}
\item \texttt{event\_name.notify(void);} \\ {\em Immediate}, the processes waiting for it are immediately moved from the waiting set into the runnable set;
\pause
\item \texttt{event\_name.notify(SC\_ZERO\_TIME);} \\ {\em Delayed}, the processes waiting for it are moved after all runnable processes in the current evaluation state have executed;
\pause
\item \texttt{event\_name.notify(sc\_time);} \\ {\em Timed}, the processes waiting for it are scheduled forward according to the given delay (greater than zero);
\pause
\item \texttt{event\_name.notify(double, sc\_time\_unit);} \\ {\em Timed}, alternative syntax.
\end{itemize}
\end{block}

\end{frame}

\begin{frame}[fragile]
\frametitle{Events}
\framesubtitle{Multiple notifications and canceling}

\begin{block}{Only the nearest notification is considered}
\vspace{-1em}
\begin{verbatim}
sc_event my_event;
my_event.notify(10, SC_NS);
my_event.notify(5, SC_NS); // only this one stays
my_event.notify(15, SC_NS);
\end{verbatim}
\vspace{-1em}
\end{block}
\pause
\begin{block}{Event notifications can be canceled}
You have to call
\begin{verbatim}
event_name.cancel();
\end{verbatim}
Note that immediate events cannot be canceled.
\end{block}

\end{frame}

\begin{frame}[fragile]
\frametitle{Events}
\framesubtitle{More complex scheduling example}

{\scriptsize 
\begin{verbatim}
sc_event action; // the action
sc_time now(sc_time_stamp()); // the current simulated time
// immediately cause action to fire
action.notify();
// schedule new action for 20 ms from now
action.notify(20, SC_MS);
// reschedule action for 1.5 ns from now
action.notify(1.5, SC_NS);
// useless, redundant
action.notify(1.5, SC_NS);
// useless, preempted by event at 1.5 ns
action.notify(3.0, SC_NS);
// reschedule action for the next evaluation step
action.notify(SC_ZERO_TIME);
// useless, preempted by action event at SC_ZERO_TIME
action.notify(1, SC_FS);
// cancel action entirely
action.cancel();
// schedule new action for 1 fs from now
action.notify(1, SC_FS);
\end{verbatim}
}
\end{frame}

\begin{frame}[fragile]
\frametitle{Events}
\framesubtitle{Waiting for an event}

\begin{block}{Full set of wait options:}
\vspace{-1em}
{\scriptsize 
\begin{verbatim}
wait(sc_time); // timeout is the event
wait(double, sc_time_unit); // convenient variant
wait(event); // single event
wait(sc_event | sc_event | ...); // any of these
wait(sc_event & sc_event & ...); // all of these
wait(sc_time, sc_event); // timeout or event
wait(sc_time, sc_event | sc_event | ...); // timeout or any event
wait(sc_time, sc_event & sc_event & ...); // timeout or all events
wait(); // static sensitivity (discussed later)
\end{verbatim}
}
\vspace{-1em}
\end{block}
\pause
\begin{block}{The fundamental rule:}
To watch for an event, a process must issue a \texttt{wait} before the corresponding \texttt{notify}.
\end{block}
\end{frame}

\begin{frame}[fragile]
\frametitle{Events}
\framesubtitle{Immediate notifications}

\begin{block}{For immediates, the order of registration counts!}
\vspace{-1em}
{\scriptsize 
\begin{verbatim}
SC_MODULE(ImmediateProblem) {
  SC_CTOR(ImmediateProblem) {
    SC_THREAD(B_thread);
    SC_THREAD(A_thread); // Problem, should go to the bottom
    SC_THREAD(C_thread);
  }
  void A_thread() {
    a_event.notify(); // Immediate
    cout << "A sent a_event!" << endl;
  }
  void B_thread() {
    wait(a_event);
    cout << "B got a_event!" << endl;
  }
  void C_thread() {
    wait(a_event);
    cout << "C got a_event!" << endl;
  }
  sc_event a_event;
};
\end{verbatim}
}
\vspace{-1em}
\end{block}
\end{frame}

\begin{frame}[fragile]
\frametitle{Events}
\framesubtitle{How to avoid missing events or deadlocks}

\begin{block}{This loops indefinitely in zero simulated time:}
\vspace{-1em}
{\tiny 
\begin{verbatim}
SC_MODULE(OrderedEvents) {
  SC_CTOR(OrderedEvents) {
    SC_THREAD(B_thread);
    SC_THREAD(A_thread);
    SC_THREAD(C_thread);
  }
  void A_thread() {
    while (true) {
      a_event.notify(SC_ZERO_TIME);
      wait(c_event);
    }
  }
  void B_thread() {
    while (true) {
      b_event.notify(SC_ZERO_TIME);
      wait(a_event);
    }
  }
  void C_thread() {
    while (true) {
      c_event.notify(SC_ZERO_TIME);
      wait(b_event);
    }
  }  
  sc_event a_event, b_event, c_event;
};
\end{verbatim}
}
\vspace{-1em}
\end{block}
\end{frame}

\begin{frame}[fragile]
\frametitle{Events}
\framesubtitle{How to avoid missing events or deadlocks}

\begin{block}{This loops indefinitely in zero simulated time:}
\vspace{-1em}
{\tiny 
\begin{verbatim}
SC_MODULE(OrderedEvents) {
  SC_CTOR(OrderedEvents) {
    SC_THREAD(B_thread);
    SC_THREAD(A_thread);
    SC_THREAD(C_thread);
  }
  void A_thread() {
    while (true) {
      a_event.notify(SC_ZERO_TIME);
      wait(c_event);
    }
  }
  void B_thread() {
    while (true) {
      b_event.notify(SC_ZERO_TIME);
      wait(a_event);
    }
  }
  void C_thread() {
    while (true) {
      c_event.notify(SC_ZERO_TIME);
      wait(b_event);
    }
  }  
  sc_event a_event, b_event, c_event;
};
\end{verbatim}
}
\vspace{-1em}
\end{block}
\end{frame}

\section{Methods}

\begin{frame}
\frametitle{Methods}
\framesubtitle{Registering a method}

\begin{block}{You register with \texttt{SC\_METHOD(method\_name);}}
Restrictions:
\begin{enumerate}
\item The same as those for \texttt{SC\_THREAD}: must be called within the constructor, and method name must take no arguments and return void;
\item No blocking calls (like \texttt{wait}) can be made within the used method: this will issue a runtime error.
\end{enumerate}
\end{block}
\pause
\begin{block}{SystemC methods vs SystemC threads}
Methods can use {\em dynamic sensitivity} to cyclically run a function, using {\em triggers}.
\end{block}
\pause
\begin{block}{SystemC methods vs class methods}
Both SystemC threads and methods run class methods.
\end{block}
\end{frame}

\begin{frame}[fragile]
\frametitle{Methods}
\framesubtitle{Triggers}

\begin{block}{These perfectly mirror the syntax for \texttt{wait}:}
\vspace{-1em}
{\scriptsize 
\begin{verbatim}
next_trigger(sc_time); // timeout is the event
next_trigger(double, sc_time_unit); // convenient variant
next_trigger(event); // single event
next_trigger(sc_event | sc_event | ...); // any of these
next_trigger(sc_event & sc_event & ...); // all of these
next_trigger(sc_time, sc_event); // timeout or event
next_trigger(sc_time, sc_event | sc_event | ...); // timeout or any 
next_trigger(sc_time, sc_event & sc_event & ...); // timeout or all
next_trigger(); // re-establish static sensitivity
\end{verbatim}
}
\vspace{-1em}
\end{block}
\pause
\begin{block}{Important points}
\vspace{-0.5em}
\begin{enumerate}
\item The trigger applies to the next execution of the method;
\item If no \texttt{next\_trigger} is issued, the method is never called again;
\item Subsequent \texttt{next\_trigger} calls override the previous ones.
\end{enumerate}
\vspace{-0.5em}
\end{block}
\end{frame}

\section{Static sensitivity}

\begin{frame}
\frametitle{Static sensitivity}
\framesubtitle{What is static sensitivity?}

\begin{block}{It is just a matter of convenience}
During module declaration, you specify which events would allow the repetition of code.
\end{block}
\pause
\begin{block}{When to use it?}
It is commonly used at the RTL (Register Transfer Logic) level.
\end{block}

\end{frame}

\begin{frame}[fragile]
\frametitle{Static sensitivity}
\framesubtitle{Declaration}

{\scriptsize
\begin{block}{{\small There exist two notations: functional and streaming}}
\vspace{0.1em}
Regardless of notation, the sensitivity applies to the {\em latest} process registration.
\begin{verbatim}
SC_MODULE(GasStation) {
  sc_event e_request1, e_request2;
  sc_event e_tank_filled;
  SC_CTOR(GasStation) {
    SC_THREAD(customer1_thread);
      sensitive(e_tank_filled); // Functional
    SC_METHOD(attendant_method);
      sensitive << e_request1 << e_request2; // Streaming
    SC_THREAD(customer2_thread);
  }
  void attendant_method();
  void customer1_thread();
  void customer2_thread();
};
\end{verbatim}
\vspace{-0.5em}
\pause
Implementation scheleton:
\begin{itemize}
\item \texttt{customer1\_thread}: use an infinite loop and issue a \texttt{wait();}
\item \texttt{customer2\_thread}: use an infinite loop and wait on an \texttt{e\_tank\_filled} event; 
\item \texttt{attendant\_method}: issue a \texttt{next\_trigger();}
\end{itemize}
\end{block}
}
\end{frame}