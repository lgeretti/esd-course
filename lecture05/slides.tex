\section{Introduction}

\begin{frame}
\frametitle{Introduction}
\framesubtitle{What is build and test automation?}

\begin{block}{Give one simple command and all is done}
\begin{itemize}
\item Reduces development time considerably
\item Is portable on different machines
\item Encourages frequent builds and tests, thus identifying issues early
\end{itemize}
\end{block}

\end{frame}

\begin{frame}
\frametitle{Introduction}
\framesubtitle{How does CMake help?}

\begin{itemize}
\item It summarizes the build and testing configuration in specific text files
\item It handles cleaning, building, testing, packaging and installing with simple commands
\item It is portable on a wide range of operating systems
\item It can generate project files for several IDEs (notably, it integrates with KDevelop)
\end{itemize}

\end{frame}

\begin{frame}
\frametitle{Introduction}
\framesubtitle{How CMake works}

\begin{itemize}
\item It holds the configuration within \texttt{CMakeLists.txt} files
\item Convention-over-configuration: anything not explicitly configured takes reasonable defaults
\item The configuration file is the real portable file: before building, it is preprocessed on a machine to generate the files that actually allow building (e.g., Makefiles)
\end{itemize}

\end{frame}

\begin{frame}
\frametitle{Introduction}
\framesubtitle{The tutorial}

\begin{block}{We will follow a tutorial hosted on BitBucket}
\begin{itemize}
\item Clone \url{https://bitbucket.org/uniud_esd/cmake_tutorial}
\item Start from the first commit and follow each commit (using the reverse notation of Git)
\end{itemize}
\end{block}
\pause
\begin{block}{We will cover the following topics:}
\begin{enumerate}
\item build sources into an executable
\item organize source folders
\item add a library
\item add a test
\item add a dependency 
\item import in an IDE
\end{enumerate}
\end{block}
\end{frame}

\section{Tutorial}

\subsection{Build}

\begin{frame}[fragile]
\frametitle{Build}
\framesubtitle{A basic configuration}

Configuration is all within a \texttt{CMakeLists.txt} file. Minimum content:

\begin{verbatim}
add_executable (execname srcfile1 srcfile2 ...)
\end{verbatim}
  
\begin{block}{Can't do less than the following:}
\begin{itemize}
\item Create a directory with a \texttt{CMakeLists.txt} file
\item In such file, choose the executable name and the list of source files to build
\end{itemize}
Comments in \texttt{CMakeLists.txt} files start with \texttt{\#}
\end{block}

\end{frame}

\begin{frame}
\frametitle{Build}
\framesubtitle{How to build}

\begin{block}{Configure and build (Linux)}
\begin{enumerate}
\item \texttt{\$ cmake .} 
\item \texttt{\$ make}
\end{enumerate}
The first command generates a Makefile, and the second uses it to build the executable. As long as \texttt{CMakeLists.txt} does not change, from now on you only need to {\em make}.
\end{block}
\pause
\begin{block}{Clean build files}
\begin{enumerate}
\item \texttt{\$ make clean}
\end{enumerate}
This still leaves some files around, including \texttt{Makefile}: these are the files needed to make.
\end{block}

\end{frame}

\subsection{Organize}

\begin{frame}
\frametitle{Organize}
\framesubtitle{Keep things tidy}

\begin{block}{Do not configure CMake on the root directory}
\begin{enumerate}
\item \texttt{\$ rm -Rf !(tutorial.c|CMakeLists.txt)}
\item \texttt{\$ mkdir build \&\& cd build}
\item \texttt{\$ cmake ..}
\item \texttt{\$ make}
\end{enumerate}
Leave the two files and configure from a build directory of your choice. Then you can \texttt{make} from there, or \\ \texttt{\$ rm -Rf *} \\ to remove CMake outputs from the build directory.
\end{block}
\pause
\begin{block}{Add a .gitignore file}
From the root directory of the project, create the file with only the directory \texttt{build} on one line.
\end{block}

\end{frame}

\begin{frame}
\frametitle{Organize}
\framesubtitle{Add a linked source file}

\begin{block}{Add the new .c source within \texttt{CMakeLists.txt}}
The corresponding .h file is not compiled separately, but only included, hence it is not considered.
\end{block}
\begin{block}{\texttt{make} builds only what changes}
You can \texttt{touch} each of the three source files to see what happens when making.
\end{block}

\end{frame}

\begin{frame}
\frametitle{Organize}
\framesubtitle{Keep sources separated}

\begin{block}{At least split .h files and .c files}
\begin{itemize}
\item The header directory is commonly called \texttt{include}, the source directory \texttt{src}
\item Header directory must be added using the \texttt{include\_directories} command
\item Source files can be simply updated with their path in the \texttt{add\_executable} command
\item We also introduced a \\
\texttt{cmake\_minimum\_required (VERSION 2.6)} \\

command to avoid a warning.
\end{itemize}
\end{block}

\end{frame}

\subsection{Libraries}

\begin{frame}
\frametitle{Libraries}
\framesubtitle{Put your function into a library}

\begin{block}{Two new commands and a modified one are needed:}
\begin{enumerate}
\item \texttt{add\_library (salut "src/salutation.c")}
\item \texttt{add\_executable (tutorial src/tutorial.c)}
\item \texttt{target\_link\_libraries (tutorial salut)}
\end{enumerate}
The first one creates a standalone library called \texttt{salut}, the second one neglects the sources used in the library, and the third one explains that the executable requires that library.
\end{block}

\end{frame}

\begin{frame}
\frametitle{Libraries}
\framesubtitle{Other libraries}

\begin{block}{Multiple libraries can be linked to an executable}
Only remember that \texttt{target\_link\_libraries} must follow both the command that adds the executable and the commands that add the libraries.
\end{block}

\begin{block}{External libraries (e.g., \texttt{pthread}) can be linked too}
Beware of portability on different operating systems, though.
\end{block}

\end{frame}

\subsection{Testing}

\begin{frame}
\frametitle{Testing}
\framesubtitle{Approach}

\begin{block}{How to test your functionality}
\begin{enumerate}
\item Functionalities are kept within libraries
\item Tests are runs of executables
\end{enumerate}
This is similar to the approach in VHDL using testbenches.
\end{block}

\end{frame}

\begin{frame}
\frametitle{Testing}
\framesubtitle{Preparation}

\begin{block}{Let's keep things separated}
\begin{itemize}
\item We move \texttt{tutorial.c} into a \texttt{test/} directory, renaming it \texttt{test\_salut.c}
\item We change the executable as \\ \texttt{add\_executable (test\_salut test/test\_salut.c)}
\end{itemize}
Summarizing, we leave the \texttt{src/} directory for library sources and the \texttt{include/} directory for the headers that users of the library would need.
\end{block}

\end{frame}

\begin{frame}
\frametitle{Testing}
\framesubtitle{Adding tests}

\begin{block}{Enable testing}
Below the executables, insert \\
\texttt{enable\_testing ()}
\end{block}

\begin{block}{Add one test}
\texttt{add\_test (CompleteWithOneArgument test\_salut you)} 

\medskip

This adds a test for \texttt{test\_salut} arbitrarily called \texttt{CompleteWithOneArgument} and an argument \texttt{you}.
\begin{itemize}
\item The executable will run with the arguments you give it
\end{itemize}
\end{block}

\end{frame}

\begin{frame}
\frametitle{Testing}
\framesubtitle{Running tests}

\texttt{\$ make test}

\bigskip

\begin{block}{Tests by default pass if}
\begin{enumerate}
\item The executable is found
\item No exception is thrown during execution
\item The executable returns zero
\end{enumerate}
This also implies that you must \texttt{make} executables before running any test that uses them!
\end{block}

\end{frame}

\begin{frame}[fragile]
\frametitle{Testing}
\framesubtitle{Tests on regular expressions}

\begin{block}{You can add as many tests as you like}
{\scriptsize
\begin{verbatim}
add_test (RightOutput test_salut me)
add_test (WrongOutput test_salut Name Surname)

set_tests_properties 
  (RightOutput PROPERTIES PASS_REGULAR_EXPRESSION "Hello, me!")
set_tests_properties 
  (WrongOutput PROPERTIES FAIL_REGULAR_EXPRESSION
    "Hello, Name Surname")
\end{verbatim}
}
Properties allow to check against text written on the standard output. See the following for POSIX regex syntax: \url{http://en.wikipedia.org/wiki/Regular_expression}
\end{block}

\end{frame}

\begin{frame}[fragile]
\frametitle{Testing}
\framesubtitle{Multiple expressions}

\begin{block}{You can in general provide multiple expressions}
{\scriptsize
\begin{verbatim}
set (passRegex "Ok" "Passed")
set (failRegex "Failed" "Error")

set_tests_properties 
  (myTest PROPERTIES PASS_REGULAR_EXPRESSION "$(passRegex)")
set_tests_properties 
  (myOtherTest PROPERTIES FAIL_REGULAR_EXPRESSION "$(failRegex)")
\end{verbatim}
}
The first test will pass if at least one of the expressions matches, while the second will pass if none of the expressions match.
\end{block}

\end{frame}

\subsection{Dependencies}

\begin{frame}
\frametitle{Dependencies}
\framesubtitle{How to add a dependency}

\begin{block}{Add the desired repo as a submodule}
\begin{enumerate}
\item {\scriptsize \texttt{\$ git submodule add git@bitbucket.org:uniud\_esd/cmake\_tutorial\_sub reverse} }
\item {\scriptsize \texttt{\$ git ci -m "Added submodule to the project"}}
\end{enumerate}
To see the content on the tutorial repo, \texttt{git submodule init} and \texttt{git submodule update} after checkout.
\end{block}
\pause
\begin{block}{Add a build subdirectory to \texttt{CMakeLists.txt}}
\begin{enumerate}
\item \texttt{add\_subdirectory ("reverse")}
\end{enumerate}
This allows to build and test everything together.
\end{block}

\end{frame}

\begin{frame}
\frametitle{Dependencies}
\framesubtitle{How to exploit a dependency}

\begin{block}{Change \texttt{test\_salut.c} to use it and have the following:}
\begin{enumerate}
\item {\scriptsize \texttt{include\_directories ("include" "reverse/include")}}
\item {\scriptsize \texttt{target\_link\_libraries (test\_salut salut reverse)}}
\end{enumerate}
The first one adds a search path for the headers of the dependency, while the second one links the corresponding library.
\begin{itemize}
\item Note that the first {\em reverse} refers to the directory, while the second one to the library name set in the submodule project
\item You also need to adapt the CMake tests to the new \texttt{test\_salut.c}.
\end{itemize}
\end{block}

\end{frame}

\subsection{Import in an IDE}

\begin{frame}
\frametitle{Import in an IDE}
\framesubtitle{Eclipse, Visual Studio, etc.}

\begin{block}{First you have to create the project files}
Generic command:

\medskip

\texttt{cmake -G "<generator>"}

\medskip

where the available generators are listed when issuing \texttt{cmake}.
\end{block}

\end{frame}

\begin{frame}
\frametitle{Import in an IDE}
\framesubtitle{KDevelop4}

\begin{block}{It allows you to directly open a \texttt{CMakeLists.txt}}
\begin{enumerate}
\item Go to \texttt{Open/Import Project}
\item Select the interested \texttt{CMakeLists.txt} file
\item Accept the default configuration it proposes to you
\end{enumerate}
KDevelop4 accounts for subdirectories and responds to changes in the \texttt{CMakeLists.txt} file.
\end{block}

\end{frame}

