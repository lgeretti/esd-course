\section{Introduction}

\begin{frame}
\frametitle{Introduction}
\framesubtitle{What is build and test automation?}

\begin{block}{Give one simple command and all is done}
\begin{itemize}
\item Reduces development time considerably
\item Is portable on different machines
\item Encourages frequent builds and tests, thus identifying issues early
\end{itemize}
\end{block}

\end{frame}

\begin{frame}
\frametitle{Introduction}
\framesubtitle{How does CMake help?}

\begin{itemize}
\item It summarizes the build and testing configuration in specific text files
\item It handles cleaning, building, testing and installing with simple commands
\item It is portable on a wide range of operating systems
\item It can generate project files for several IDEs (notably, it integrates with KDevelop)
\end{itemize}

\end{frame}

\begin{frame}
\frametitle{Introduction}
\framesubtitle{The tutorial}

\begin{block}{We will follow a tutorial hosted on BitBucket}
\begin{itemize}
\item Clone \url{https://bitbucket.org/uniud_esd/cmake_tutorial}
\item Start from the first commit and follow each commit
\item We will cover the following topics:
\begin{enumerate}
\item build sources into an executable
\item organize source folders
\item add a library
\item add a dependency 
\item add a test
\end{enumerate}
\end{itemize}
\end{block}
\end{frame}

\section{Tutorial}

\subsection{Build}

\begin{frame}[fragile]
\frametitle{Build}
\framesubtitle{A basic configuration}

Configuration is all within a \texttt{CMakeLists.txt} file. Minimum content:

\begin{verbatim}
add_executable (execname sourcefile1 sourcefile2)
\end{verbatim}
  
\begin{block}{Can't do less than the following:}
\begin{itemize}
\item Create a directory with a \texttt{CMakeLists.txt} file
\item In such file, choose the executable name and the list of source files to build
\end{itemize}
Comments in \texttt{CMakeLists.txt} files start with \texttt{\#}
\end{block}

\end{frame}

\begin{frame}
\frametitle{Build}
\framesubtitle{How to build}

\begin{block}{Configure and build (Linux)}
\begin{enumerate}
\item \texttt{\$ cmake .} 
\item \texttt{\$ make}
\end{enumerate}
The first command generates a Makefile, and the second uses it to build the executable. As long as \texttt{CMakeLists.txt} does not change, from now on you only need to {\em make}.
\end{block}
\pause
\begin{block}{Clean build files}
\begin{enumerate}
\item \texttt{\$ make clean}
\end{enumerate}
This still leaves some files around, including \texttt{Makefile}: these are the files needed to make.
\end{block}

\end{frame}

\begin{frame}
\frametitle{Organize}
\framesubtitle{Keep things tidy}

\begin{block}{Do not configure CMake on the root directory}
\begin{enumerate}
\item \texttt{\$ rm -Rf !(tutorial.c|CMakeLists.txt)}
\item \texttt{\$ mkdir build \&\& cd build}
\item \texttt{\$ cmake ..}
\item \texttt{\$ make}
\end{enumerate}
Leave the two files and configure from a build directory of your choice. Then you can make from there.
\end{block}
\pause
\begin{block}{Add a .gitignore file}
From the root directory of the project, create the file with only the directory \texttt{build} on one line.
\end{block}

\end{frame}

\begin{frame}
\frametitle{Organize}
\framesubtitle{Add a linked source file}

\begin{block}{Add the new .c source within \texttt{CMakeLists.txt}}
The corresponding .h file is not compiled separately, but only included, hence it is not considered.
\end{block}
\begin{block}{\texttt{make} builds only what changes}
You can \texttt{touch} each of the three source files to see what happens when making.
\end{block}

\end{frame}

\subsection{Organize}

\begin{frame}
\frametitle{Organize}
\framesubtitle{Keep sources separated}

\begin{block}{At least split .h files and .c files}
\begin{itemize}
\item The header directory is commonly called \texttt{include}, the source directory \texttt{src}
\item Header directory must be added using the \texttt{include\_directories} command
\item Source files can be simply updated with their path in the \texttt{add\_executable} command
\item We also added a \\
\texttt{cmake\_minimum\_required (VERSION 2.6)} \\

command to avoid a warning.

\end{itemize}
\end{block}

\end{frame}